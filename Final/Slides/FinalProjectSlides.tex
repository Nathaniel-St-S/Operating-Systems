\documentclass[aspectratio=169]{beamer}
\setbeamertemplate{footline}{}
\setbeamertemplate{navigation symbols}{}

% ---------------------------------------------------------
% Nord Color Palette
% ---------------------------------------------------------
\definecolor{Nord0}{HTML}{2E3440} % #2E3440
\definecolor{Nord1}{HTML}{3B4252} % #3B4252
\definecolor{Nord2}{HTML}{434C5E} % #434C5E
\definecolor{Nord3}{HTML}{4C566A} % #4C566A
\definecolor{Nord4}{HTML}{D8DEE9} % #D8DEE9
\definecolor{Nord5}{HTML}{E5E9F0} % #E5E9F0
\definecolor{Nord6}{HTML}{ECEFF4} % #ECEFF4
\definecolor{Nord7}{HTML}{8FBCBB} % #8FBCBB
\definecolor{Nord8}{HTML}{88C0D0} % #88C0D0
\definecolor{Nord9}{HTML}{81A1C1} % #81A1C1
\definecolor{Nord10}{HTML}{5E81AC} % #5E81AC
\definecolor{Nord11}{HTML}{BF616A} % #BF616A
\definecolor{Nord12}{HTML}{D08770} % #D08770
\definecolor{Nord13}{HTML}{EBCB8B} % #EBCB8B
\definecolor{Nord14}{HTML}{A3BE8C} % #A3BE8C
\definecolor{Nord15}{HTML}{B48EAD} % #B48EAD

% ---------------------------------------------------------
% Theme Setup
% ---------------------------------------------------------
\usetheme{default}

% Background + text
\setbeamercolor{background canvas}{bg=Nord0}
\setbeamercolor{normal text}{fg=Nord6}

% Frame titles
\setbeamercolor{frametitle}{bg=Nord0, fg=Nord8}
\setbeamerfont{frametitle}{size=\large,series=\bfseries}

% Section titles
\setbeamercolor{section title}{fg=Nord11}

% Itemize
\setbeamercolor{itemize item}{fg=Nord9}
\setbeamercolor{itemize subitem}{fg=Nord10}

% Blocks
\setbeamercolor{block title}{bg=Nord2, fg=Nord6}
\setbeamercolor{block body}{bg=Nord1, fg=Nord6}

% Links
\hypersetup{
    colorlinks=true,
    urlcolor=Nord8,
    linkcolor=Nord14
}

% Packages
\usepackage{listings}
\usepackage{xcolor}
\definecolor{pakistangreen}{RGB}{0,102,0}
\definecolor{oceanboatblue}{RGB}{0,119,190}
\definecolor{vscString}{RGB}{152,195,121}
\definecolor{vscType}{RGB}{86,182,194}
\definecolor{vscPreproc}{RGB}{198,120,221}
\definecolor{neonfuchsia}{RGB}{254, 89, 194}
%defualt style for C listings
\lstset{
	% --- Linenos ---
	numbersep=3pt,                  % how far the line-numbers are from the code
	numberstyle=\tiny\color{gray},
	numbers=left,                   % where to put the line-numbers
	stepnumber=1,                   % the step between two line-numbers. If it is 1 each line will be numbered
	% -- Basics --
	basicstyle=\ttfamily\small,
	sensitive=true,  % Case-sensitive keywords
	tabsize=2,
	breaklines=true,  % Break lines if too long
	columns=fullflexible,
	keepspaces,
	showstringspaces=false,  % Spaces not shown as _
	upquote=true,
	%keywordstyle={\color{purple}}
}

\lstset{numbers=left,xleftmargin=2em,framexleftmargin=1.5em}

% Usage: \begin{lstlisting}[language=C] ...
	\lstdefinelanguage{MyC} {
		% -- Comments --
		morecomment=[l]{//},
		morecomment=[s]{/*}{*/},
		%morecomment=[l]{;},         % Inline comments start with ;
		%morecomment=[s]{\#|}{|\#},  % Block comments are done with #|  |#
		commentstyle={\color{pakistangreen}\slshape\sffamily},
		% -- Strings --
		morestring=[b]",
		stringstyle=\color{vscString},
		% --- Literal replacements ---
		literate=
		{\$}{\$}1,
		%{*}{{\color{cyan}{*}}}1
		%{&}{\color{blue}{&}}1
		%{&&}{{\color{cyan}{&&}}}1,
		%{||}{{\color{cyan}{||}}}1,
		%	{lambda}{{\$\lambda\$}}1
		%{->}{{\$\rightarrow\$}}1
		%{*}{{*}}1
		%{lambda}{{\textcolor{blue}{\(\lambda\)}}}1
		%{EMP}{{\ep{}}}1,
		%{'}{{\quot{}}}1,
		classoffset=1,
		morekeywords={
			void, main, typedef, struct, enum, static, int, float, double, bool, true, false, stderr, NULL, sizeof, const, char, unsigned, long, inline, printf, fprintf
		},
		keywordstyle=\color{oceanboatblue},
		classoffset=2,
		morekeywords={
			 if, for, while, else, switch, return, break, case, default 
		},
		keywordstyle=\color{neonfuchsia},
		classoffset=3,
		morekeywords={
			import, export
			},
		keywordstyle=\color{yellow},
		classoffset=4,
		morekeywords={
			Cpu, ProcessState, uint32_t, size_t, Process, Queue, MemoryBlock, Cache, CachePolicy, uint8_t, CacheLine, MemoryTable, uint16_t, int32_t, uint64_t, int64_t, AssemblyContext, int8_t, int16_t, DataSegment, TextEntry, Macro, Symbol, AssemblyResult, AssembledProgram, SymbolInfo, PerformanceTracker, ProcessMetrics, PerformanceMetrics, timespec, PerfTimer, SchedulingAlgorithm, IRQ, InterruptHeap, Interrupt, stack, Entry, dword, word, hword
			},
		keywordstyle=\color{palatinateblue},
		classoffset=5,
		morekeywords={
			define
		},
		keywordstyle=\color{vscPreproc},
		%classoffset=0,
		alsoletter={',`,-,/,>,<,\#,\$,?,=,&&},
		%moredelim=**[is][\color{lightgray}]{<<@<<}{>>@>>},
		%moredelim=**[is][\itshape\color{OliveGreen}]{<<;<<}{>>;>>},
	}
	% End of C listing style 
\usepackage{fontspec}
\setmonofont{JetBrainsMono Nerd Font}


% ---------------------------------------------------------
% Title Info
% ---------------------------------------------------------
\title{Operating Systems Final Project}
\subtitle{Advanced OS Simulator}
\author{David Fields \and Brysen Pfingsten \and Nathanial Savoury}
\institute{Seton Hall Univeristy}
\date{Fall 2025}

% ---------------------------------------------------------
% Document
% ---------------------------------------------------------
\begin{document}
\AtBeginSection[]
{
  \begin{frame}
    \centering
    {%
      \hypersetup{hidelinks}%
      {\usebeamercolor[fg]{section title}\Huge\insertsection}%
    }
  \end{frame}
}
% ---------- TITLE SLIDE ----------
{
\setbeamercolor{title}{fg=Nord8}
\begin{frame}
    \titlepage
\end{frame}
}

% ---------- TABLE OF CONTENTS ----------
\begin{frame}{Outline}
    \tableofcontents
\end{frame}

%================================================
\section{Module 1: Process Simulation}
%================================================

\begin{frame}[fragile]{CPU Definition}
\begin{lstlisting}[
  language=MyC,
  basicstyle=\fontsize{9pt}{7pt}\selectfont\ttfamily,
  aboveskip=0pt,
  belowskip=0pt
]
typedef struct Cpu {
  uint32_t gp_registers[GP_REG_COUNT];
  uint32_t hw_registers[HW_REG_COUNT];
} Cpu;

// General Purpose Registers
enum {
  REG_ZERO, REG_AT, REG_VO, REG_V1,
  REG_A0, REG_A1, REG_A2, REG_A3,
  REG_T0, REG_T1, REG_T2, REG_T3, REG_T4, REG_T5, REG_T6, REG_T7,
  REG_S0, REG_S1, REG_S2, REG_S3, REG_S4, REG_S5, REG_S6, REG_S7,
  REG_T8, REG_T9, REG_K0, REG_K1,
  REG_GP, REG_SP, REG_FP, REG_RA,
  GP_REG_COUNT,
};

// Hardware Registers
enum {
  PC, IR, MAR, MBR,
  IO_AR, IO_BR,
  FLAGS, HI, LO,
  HW_REG_COUNT,
};
\end{lstlisting}
\end{frame}

\begin{frame}[fragile]{Process Control Block}
\begin{lstlisting}[
  language=MyC,
  basicstyle=\fontsize{9pt}{7pt}\selectfont\ttfamily,
  aboveskip=0pt,
  belowskip=0pt
]
typedef struct {
  int pid;
  uint32_t pc;
  ProcessState state;
  int priority;
  int burstTime, originalBurstTime;
  float responseRatio;

  Cpu cpu_state;

  uint32_t text_start,  text_size;
  uint32_t data_start,  data_size;
  uint32_t stack_ptr;

  // Performance tracking
  int arrival_time;
  int start_time;
  int completion_time;
  int waiting_time;
  int response_time;
  bool has_started;
} Process;
\end{lstlisting}
\end{frame}


\begin{frame}{MIPS-I Subset 32-bit ISA Overview}
\small

\textbf{What this ISA is}
\begin{itemize}
  \item 32-bit load/store RISC ISA based closely on the original MIPS~I design.
  \item All instructions are fixed 32 bits wide.
\end{itemize}

\medskip

\textbf{Registers}
\begin{itemize}
  \item 32 general-purpose registers (GPRs), each 32 bits wide.
  \item Special registers: \texttt{PC} (program counter), \texttt{HI}, \texttt{LO}.
\end{itemize}

\medskip

\textbf{Instruction Formats}
\begin{itemize}
  \item Three formats: \textbf{R-type}, \textbf{I-type}, and \textbf{J-type}.
  \item All instructions follow one of these three encodings.
\end{itemize}
\end{frame}

\begin{frame}{Instruction Types}
\scriptsize

\textbf{R-Type Format}
\begin{center}
\begin{tabular}{|c|c|c|c|c|c|}
\hline
31--26 & 25--21 & 20--16 & 15--11 & 10--6 & 5--0 \\ \hline
\texttt{opcode} & \texttt{rs} & \texttt{rt} & \texttt{rd} & \texttt{shamt} & \texttt{funct} \\ \hline
6 bits & 5 bits & 5 bits & 5 bits & 5 bits & 6 bits \\ \hline
\end{tabular}
\end{center}

\[
\texttt{opcode} = 0 \text{ for all R-Type instructions}
\]

\vspace{0.4em}

\textbf{I-Type Format}
\begin{center}
\begin{tabular}{|c|c|c|c|}
\hline
31--26 & 25--21 & 20--16 & 15--0 \\ \hline
\texttt{opcode} & \texttt{rs} & \texttt{rt} & \texttt{immediate} \\ \hline
6 bits & 5 bits & 5 bits & 16 bits \\ \hline
\end{tabular}
\end{center}

\begin{center}
\footnotesize Immediate is sign-extended or zero-extended depending on the instruction.
\end{center}

\vspace{0.4em}

\textbf{J-Type Format}
\begin{center}
\begin{tabular}{|c|c|}
\hline
31--26 & 25--0 \\ \hline
\texttt{opcode} & \texttt{target} \\ \hline
6 bits & 26 bits \\ \hline
\end{tabular}
\end{center}

\[
\text{PC}_{\text{next}}
= (\text{PC}_{\text{current}}[31{:}28] \ll 28)
\;|\; (\texttt{target} \ll 2)
\]

\end{frame}



\begin{frame}[fragile]{Opcode and Funct Codes}
\begin{lstlisting}[
  language=MyC,
  basicstyle=\fontsize{6pt}{7pt}\selectfont\ttfamily,
  aboveskip=0pt,
  belowskip=0pt,
]
enum { 
  OP_ADD=0x0,  OP_ADDU=0x0, OP_SUB=0x0,  OP_SUBU=0x0,
  OP_MULT=0x0, OP_MULTU=0x0, OP_DIV=0x0, OP_DIVU=0x0,
  OP_MFHI=0x0, OP_MFLO=0x0, OP_MTHI=0x0, OP_MTLO=0x0,
  OP_AND=0x0,  OP_OR=0x0,   OP_XOR=0x0,  OP_NOR=0x0,
  OP_SLL=0x0,  OP_SRL=0x0,  OP_SRA=0x0,
  OP_SLLV=0x0, OP_SRLV=0x0, OP_SRAV=0x0,

  OP_ADDI=0x08,  OP_ADDIU=0x09, OP_ANDI=0x0C, OP_ORI=0x0D,
  OP_XORI=0x0E, OP_SLTI=0x0A,   OP_SLTIU=0x0B, OP_LUI=0x0F,

  OP_LW=0x23, OP_SW=0x2B, OP_LB=0x20, OP_LBU=0x24,
  OP_LH=0x21, OP_LHU=0x25, OP_SB=0x28, OP_SH=0x29,

  OP_BEQ=0x04, OP_BNE=0x05, OP_J=0x02, OP_JAL=0x03,
  OP_JR=0x0,   OP_JALR=0x0, OP_SYSCALL=0x0, OP_BREAK=0x0,
  OP_ERET=0x10
};

enum {
  FUNCT_ADD=0x20, FUNCT_ADDU=0x21, FUNCT_SUB=0x22, FUNCT_SUBU=0x23,
  FUNCT_MULT=0x18, FUNCT_MULTU=0x19, FUNCT_DIV=0x1A, FUNCT_DIVU=0x1B,
  FUNCT_MFHI=0x10, FUNCT_MFLO=0x12, FUNCT_MTHI=0x11, FUNCT_MTLO=0x13,

  FUNCT_AND=0x24, FUNCT_OR=0x25, FUNCT_XOR=0x26, FUNCT_NOR=0x27,
  FUNCT_SLL=0x00, FUNCT_SRL=0x02, FUNCT_SRA=0x03,
  FUNCT_SLLV=0x04, FUNCT_SRLV=0x06, FUNCT_SRAV=0x07,

  FUNCT_JR=0x08, FUNCT_JALR=0x09,
  FUNCT_SYSCALL=0x0C, FUNCT_BREAK=0x0D
};
\end{lstlisting}
\end{frame}

\begin{frame}{Instruction Set Architecture (ISA)}
  % ISA
\end{frame}

%------------------------------------------------
\section{Module 2: Memory Management}
%------------------------------------------------

\begin{frame}{Memory Hierarchy}
  % L1, L2, RAM, Disk
\end{frame}

\begin{frame}{Cache Lookup}
  % Cache Lookup
\end{frame}

\begin{frame}{Cache Policies}
  % Write through and write back
\end{frame}

\begin{frame}{Memory Table}
  % Memory Table
\end{frame}

\begin{frame}{Memory Allocation}
  % Allocation
\end{frame}

\begin{frame}{Memory Deallocation}
  % Deallocation
\end{frame}

%================================================
\section{Module 3: Process Scheduling and Context Switching}
%================================================

\subsection{Scheduling Algorithms}

\begin{frame}{Round Robin}
  % Round Robin
\end{frame}

\begin{frame}{Priority Based Scheduling}
  % Priority Based Scheduling
\end{frame}

\begin{frame}{Shortest Remaining Time}
  % Shortest Remaining Time
\end{frame}

\begin{frame}{Highest Response Ratio Next}
  % Highest Response Ratio Next
\end{frame}

\begin{frame}{First-Come First-Served (FCFS)}
  % FCFS
\end{frame}

\begin{frame}{Shortest Process Next}
  % Shortest Process Next
\end{frame}

\begin{frame}{Feedback Scheduling Algorithm}
  % Feedback Scheduling Algorithm
\end{frame}

\begin{frame}{Context Switching}
  % Context Switching
\end{frame}

\begin{frame}{Integration with Fetch-Decode-Execute}
  % Integration with Fetch-Decode-Execute
\end{frame}

%================================================
\section{Module 4: Enhanced Interrupt Handling}
%================================================

\begin{frame}{Interrupts}
  % Interrupts
\end{frame}

\begin{frame}{Interrupt Vector Table (IVT)}
  % Interrupt Vector Table (IVT)
\end{frame}

\begin{frame}{Interrupt Handlers and Dispatcher}
  % Interrupt Handlers and Dispatcher
\end{frame}

%================================================
\section{Module 5: Efficiency Analysis of Scheduling Algorithms}
%================================================

\begin{frame}{Execution Time}
  % Execution Time
\end{frame}

\begin{frame}{CPU Utilisation}
  % CPU Utilisation
\end{frame}

\begin{frame}{Memory Utilisation}
  % Memory Utilisation
\end{frame}

\begin{frame}{Data Collection for Comparison}
  % Data Collection for Comparison
\end{frame}

\begin{frame}{Comparing Scheduling Algorithms}
  % Compare scheduling algorithms
\end{frame}

\begin{frame}{Graphs and Visualisation}
  % Graphs
\end{frame}
\end{document}
