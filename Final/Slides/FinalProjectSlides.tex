\documentclass[aspectratio=169]{beamer}
\setbeamertemplate{footline}{}
\setbeamertemplate{navigation symbols}{}

% ---------------------------------------------------------
% Nord Color Palette
% ---------------------------------------------------------
\definecolor{Nord0}{HTML}{2E3440} % #2E3440
\definecolor{Nord1}{HTML}{3B4252} % #3B4252
\definecolor{Nord2}{HTML}{434C5E} % #434C5E
\definecolor{Nord3}{HTML}{4C566A} % #4C566A
\definecolor{Nord4}{HTML}{D8DEE9} % #D8DEE9
\definecolor{Nord5}{HTML}{E5E9F0} % #E5E9F0
\definecolor{Nord6}{HTML}{ECEFF4} % #ECEFF4
\definecolor{Nord7}{HTML}{8FBCBB} % #8FBCBB
\definecolor{Nord8}{HTML}{88C0D0} % #88C0D0
\definecolor{Nord9}{HTML}{81A1C1} % #81A1C1
\definecolor{Nord10}{HTML}{5E81AC} % #5E81AC
\definecolor{Nord11}{HTML}{BF616A} % #BF616A
\definecolor{Nord12}{HTML}{D08770} % #D08770
\definecolor{Nord13}{HTML}{EBCB8B} % #EBCB8B
\definecolor{Nord14}{HTML}{A3BE8C} % #A3BE8C
\definecolor{Nord15}{HTML}{B48EAD} % #B48EAD

% ---------------------------------------------------------
% Theme Setup
% ---------------------------------------------------------
\usetheme{default}

% Background + text
\setbeamercolor{background canvas}{bg=Nord0}
\setbeamercolor{normal text}{fg=Nord6}

% Frame titles
\setbeamercolor{frametitle}{bg=Nord0, fg=Nord8}
\setbeamerfont{frametitle}{size=\large,series=\bfseries}

% Section titles
\setbeamercolor{section title}{fg=Nord8}

% Itemize
\setbeamercolor{itemize item}{fg=Nord9}
\setbeamercolor{itemize subitem}{fg=Nord10}

% Blocks
\setbeamercolor{block title}{bg=Nord2, fg=Nord6}
\setbeamercolor{block body}{bg=Nord1, fg=Nord6}

% Links
\hypersetup{
    colorlinks=true,
    urlcolor=Nord8,
    linkcolor=Nord14
}

% Packages
\usepackage{listings}
\usepackage{xcolor}
\definecolor{pakistangreen}{RGB}{0,102,0}
\definecolor{oceanboatblue}{RGB}{0,119,190}
\definecolor{vscString}{RGB}{152,195,121}
\definecolor{vscType}{RGB}{86,182,194}
\definecolor{vscPreproc}{RGB}{198,120,221}
%defualt style for C listings
\lstset{
	% --- Linenos ---
	numbersep=3pt,                  % how far the line-numbers are from the code
	numberstyle=\tiny\color{gray},
	numbers=left,                   % where to put the line-numbers
	stepnumber=1,                   % the step between two line-numbers. If it is 1 each line will be numbered
	% -- Basics --
	basicstyle=\ttfamily\small,
	sensitive=true,  % Case-sensitive keywords
	tabsize=2,
	breaklines=true,  % Break lines if too long
	columns=fullflexible,
	keepspaces,
	showstringspaces=false,  % Spaces not shown as _
	upquote=true,
	%keywordstyle={\color{purple}}
}

\lstset{numbers=left,xleftmargin=2em,framexleftmargin=1.5em}

% Usage: \begin{lstlisting}[language=C] ...
	\lstdefinelanguage{MyC} {
		% -- Comments --
		morecomment=[l]{//},
		morecomment=[s]{/*}{*/},
		%morecomment=[l]{;},         % Inline comments start with ;
		%morecomment=[s]{\#|}{|\#},  % Block comments are done with #|  |#
		commentstyle={\color{pakistangreen}\slshape\sffamily},
		% -- Strings --
		morestring=[b]",
		stringstyle=\color{vscString},
		% --- Literal replacements ---
		literate=
		{\$}{\$}1,
		%{*}{{\color{cyan}{*}}}1
		%{&}{\color{blue}{&}}1
		%{&&}{{\color{cyan}{&&}}}1,
		%{||}{{\color{cyan}{||}}}1,
		%	{lambda}{{\$\lambda\$}}1
		%{->}{{\$\rightarrow\$}}1
		%{*}{{*}}1
		%{lambda}{{\textcolor{blue}{\(\lambda\)}}}1
		%{EMP}{{\ep{}}}1,
		%{'}{{\quot{}}}1,
		classoffset=1,
		morekeywords={
			void, main, typedef, struct, enum, static, int, float, double, bool, true, false, stderr, NULL, sizeof, const, char, unsigned, long, inline, printf, fprintf
		},
		keywordstyle=\color{oceanboatblue},
		classoffset=2,
		morekeywords={
			 if, for, while, else, switch, return, break, case, default 
		},
		keywordstyle=\color{neonfuchsia},
		classoffset=3,
		morekeywords={
			import, export
			},
		keywordstyle=\color{yellow},
		classoffset=4,
		morekeywords={
			Cpu, ProcessState, uint32_t, size_t, Process, Queue, MemoryBlock, Cache, CachePolicy, uint8_t, CacheLine, MemoryTable, uint16_t, int32_t, uint64_t, int64_t, AssemblyContext, int8_t, int16_t, DataSegment, TextEntry, Macro, Symbol, AssemblyResult, AssembledProgram, SymbolInfo, PerformanceTracker, ProcessMetrics, PerformanceMetrics, timespec, PerfTimer, SchedulingAlgorithm, IRQ, InterruptHeap, Interrupt, stack, Entry, dword, word, hword
			},
		keywordstyle=\color{palatinateblue},
		classoffset=5,
		morekeywords={
			define
		},
		keywordstyle=\color{vscPreproc},
		%classoffset=0,
		alsoletter={',`,-,/,>,<,\#,\$,?,=,&&},
		%moredelim=**[is][\color{lightgray}]{<<@<<}{>>@>>},
		%moredelim=**[is][\itshape\color{OliveGreen}]{<<;<<}{>>;>>},
	}
	% End of C listing style 


% ---------------------------------------------------------
% Title Info
% ---------------------------------------------------------
\title{Operating Systems Final Project}
\subtitle{Advanced OS Simulator}
\author{David Fields \and Brysen Pfingsten \and Nathanial Savoury}
\institute{Seton Hall Univeristy}
\date{Fall 2025}

% ---------------------------------------------------------
% Document
% ---------------------------------------------------------
\begin{document}

% ---------- TITLE SLIDE ----------
{
\setbeamercolor{title}{fg=Nord8}
\begin{frame}
    \titlepage
\end{frame}
}

% ---------- TABLE OF CONTENTS ----------
\begin{frame}{Outline}
    \tableofcontents
\end{frame}

% ========= 1. INTRO =========
\section{Introduction}

\begin{frame}{Project Overview}
    \begin{itemize}
        \item Advanced OS simulator in C that assembles multiple MIPS-like programs, loads them into isolated text/data/stack regions, and schedules them under different policies.
        \item End-to-end pipeline: parse CLI options (\texttt{main.c}), assemble programs, allocate memory with selectable cache policy, create PCBs, then run the chosen scheduler.
        \item Metrics captured per algorithm: waiting/turnaround/response time, context switches, CPU utilization, cache stats, and CSV exports for charting.
    \end{itemize}
\end{frame}

% ========= 2. ARCHITECTURE =========
\section{Simulator Architecture}

\begin{frame}{High-Level Architecture}
    \begin{itemize}
        \item \textbf{Execution pipeline}: \texttt{main.c} parses flags (scheduler + cache policy), assembles programs, calls \texttt{init\_memory}/\texttt{init\_queues}, builds PCBs with \texttt{makeProcess()}, then invokes \texttt{scheduler()}.
        \item \textbf{CPU + ISA}: fetch/execute loop in \texttt{cpu.c}/\texttt{isa.c} drives instructions from RAM; register files are plain arrays for GP/HW regs and FLAGS.
        \item \textbf{Memory system}: 128\,MB RAM backed by 64\,B L1 and 128\,B L2 caches (64\,B lines), plus SSD/HDD buffers; runtime switchable between write-through and write-back.
        \item \textbf{Scheduling layer}: ready/running/blocked queues feed FCFS, RR, Priority, SRT, SPN, HRRN, and MLFQ; scheduler swaps \texttt{THE\_CPU} state between PCBs and logs performance metrics.
    \end{itemize}
\end{frame}

\begin{frame}[fragile]{Core Data Structures}
    \begin{itemize}
        \item \textbf{PCB (\texttt{Process} in \texttt{processes.c})}: tracks PID, PC, priority/burst estimates, response ratio, full CPU register snapshot, text/data/stack bounds, and timing stats for reporting.
    \end{itemize}
\vspace{0.4em}
\begin{lstlisting}[language=MyC]
typedef struct {
  int pid;
  uint32_t pc;
  ProcessState state;
  int priority, burstTime, originalBurstTime;
  float responseRatio;
  Cpu cpu_state;                 // saved registers
  uint32_t text_start, text_size;
  uint32_t data_start, data_size;
  uint32_t stack_ptr;
  int arrival_time, start_time;
  int completion_time, waiting_time, response_time;
  bool has_started;
} Process;
\end{lstlisting}
\vspace{0.4em}
    \begin{itemize}
        \item \textbf{Queues}: each ready/blocked/suspend queue is a fixed-capacity \texttt{Queue} with an in-place PCB array; priority and burst-based heaps are used for SPN/SRT/Priority.
        \item \textbf{Memory metadata}: \texttt{CacheLine} (tag/valid/dirty + 64\,B payload) powers L1/L2, while \texttt{MemoryBlock}/\texttt{MemoryTable} track best-fit allocations per PID with 1\,MB virtual spacing.
        \item \textbf{Performance tracker}: \texttt{PerformanceMetrics} aggregates per-algorithm latency stats, cache hits/misses, context-switch timings, and exports comparison CSVs.
    \end{itemize}
\end{frame}

% ========= 3. MODULE 1 =========
\section{Module 1: Process Simulation}

\begin{frame}{CPU Model \& Registers}
    \begin{itemize}
        \item Implemented registers: PC, ACC, IR, Status Register.
        \item Fetch–Decode–Execute cycle in the simulator.
        \item How you represent instructions (e.g., struct, enum, opcodes).
    \end{itemize}
\end{frame}

\begin{frame}{Process Control Block \& States}
    \begin{itemize}
        \item PCB fields and why they are needed.
        \item Process states: READY, RUNNING, BLOCKED (and FINISHED if used).
        \item Example: life cycle of a process in your simulator.
    \end{itemize}
\end{frame}

% ========= 4. MODULE 2 =========
\section{Module 2: Memory Management}

\begin{frame}{Memory Hierarchy Design}
    \begin{itemize}
        \item L1 cache: size, role, typical hit rate (if measured).
        \item L2 cache: relation to L1 and RAM.
        \item RAM layout for processes and code/data.
    \end{itemize}
\end{frame}

\begin{frame}{Memory Table \& Allocation}
    \begin{itemize}
        \item Structure of the memory table.
        \item Implemented strategies: First-Fit, Best-Fit.
        \item Example allocation / deallocation scenario.
    \end{itemize}
\end{frame}

\begin{frame}{Cache Policies}
    \begin{itemize}
        \item Read path: L1 → L2 → RAM.
        \item Write-through vs write-back (which you implemented / compared).
        \item Any interesting cache hit/miss behavior you observed.
    \end{itemize}
\end{frame}

% ========= 5. MODULE 3 =========
\section{Module 3: Scheduling \& Context Switching}

\begin{frame}{Implemented Scheduling Algorithms}
    \begin{itemize}
        \item FCFS, Round Robin, Priority-based.
        \item Shortest Remaining Time (SRT), SPN, HRRN.
        \item Feedback (multi-level queue) scheduling.
    \end{itemize}
\end{frame}

\begin{frame}{Context Switching \& Integration}
    \begin{itemize}
        \item \texttt{contextSwitch(from, to)}: what gets saved and restored.
        \item Integration with fetch–decode–execute cycle (time slice checks, preemption).
        \item Handling CPU-bound vs I/O-bound processes.
    \end{itemize}
\end{frame}

% ========= 6. MODULE 4 =========
\section{Module 4: Interrupt Handling}

\begin{frame}{Interrupt Types \& IVT}
    \begin{itemize}
        \item Timer, I/O, System Call, Trap interrupts.
        \item Structure of the Interrupt Vector Table.
        \item How an interrupt is detected in the simulator.
    \end{itemize}
\end{frame}

\begin{frame}{ISRs \& Dispatcher}
    \begin{itemize}
        \item Steps inside an ISR: save state, handle event, signal scheduler.
        \item Dispatcher role after an interrupt: choosing next process.
        \item Example: timer interrupt causing a context switch.
    \end{itemize}
\end{frame}

% ========= 7. MODULE 5 =========
\section{Module 5: Performance Analysis}

\begin{frame}{Metrics and Workloads}
    \begin{itemize}
        \item Metrics: execution time, average waiting time, turnaround time.
        \item CPU utilization and (optionally) memory/cache utilization.
        \item Workloads: CPU-bound, I/O-bound, mixed; sample process set.
    \end{itemize}
\end{frame}

\begin{frame}{Scheduling Results (Table)}
    \begin{itemize}
        \item Present table: Algorithm vs Avg Waiting Time vs Avg Turnaround vs CPU usage.
        \item Highlight a couple of key numbers (best/worst algorithms).
        \item Mention how measurements were obtained (e.g., \texttt{clock()}).
    \end{itemize}
    % Insert screenshot or table here
\end{frame}

\begin{frame}{Graphs \& Observations}
    \begin{itemize}
        \item Bar/line charts comparing algorithms (waiting time, turnaround).
        \item Which algorithm works best for CPU-bound vs I/O-bound vs mixed.
        \item Trade-offs: fairness vs response time vs complexity.
    \end{itemize}
\end{frame}

% ========= 8. DEMO PLAN =========
\section{Live Demo}

\begin{frame}{Demo Scenario}
    \begin{itemize}
        \item Input: small set of processes (arrival, burst, priority).
        \item Show: scheduler switching, context switch messages, interrupt events.
        \item What you want the professor to notice in the demo (e.g., differences between algorithms).
    \end{itemize}
\end{frame}

% ========= 9. CONCLUSION =========
\section{Conclusion}

\begin{frame}{Challenges \& Debugging}
    \begin{itemize}
        \item Major bugs/issues you faced (race conditions, incorrect metrics, etc.).
        \item How you tested and validated correctness.
    \end{itemize}
\end{frame}

\begin{frame}{Key Learnings \& Future Work}
    \begin{itemize}
        \item What you learned about OS internals and system-level programming.
        \item Limitations of the current simulator.
        \item Possible extensions (e.g., more realistic I/O, threads, new policies).
    \end{itemize}
\end{frame}

\begin{frame}{Q \& A}
    \centering
    \Large Questions?
\end{frame}

\end{document}
