\section{Module 2: Advanced Memory Management}
\label{Memory-Spec}

\Update{}

\subsection{Problem Statement}
\label{PS-m2}
In this module, you’ll expand the memory management system from Project 2 by adding features for efficient memory allocation, hierarchical memory structure, dynamic memory allocation, and shared memory access. Compared to Project 3, this module introduces enhanced features for memory efficiency and concurrency, simulating a more realistic and sophisticated operating system environment.

\subsection{Implementation}
\label{IMP-m2}
\subsubsection{Hierarchical Memory System}
To simulate a hierarchical memory system, we implemented cache lines which stores the data in an array to allow for multiple sizes of data without overwriting data. The cache line is represented as structure:
\begin{lstlisting} [language=MyC, escapechar=\$, numbers=none]
typedef struct {
	uint32_t tag;
	bool is_valid;
	bool is_dirty;
	uint8_t data[CACHE_LINE_SIZE];
} CacheLine;
\end{lstlisting}

The L1 and L2 caches are structures: 

\begin{lstlisting}[language=MyC, escapechar=\$, numbers=none]
	typedef struct {
		CacheLine *lines;
		size_t front;
		size_t count;
		size_t line_count;
	} Cache;
\end{lstlisting}
%
A \texttt{MemoryBlock} is a structure: 
%
\begin{lstlisting}[language=MyC, escapechar=\$, numbers=none]
	typedef struct {
		int pid;
		uint32_t start_addr;
		uint32_t end_addr;
		bool is_free;
	} MemoryBlock;
\end{lstlisting}

\begin{lstlisting}[language=MyC, escapechar=\$, numbers=none]
	typedef struct {
		MemoryBlock *blocks;
		size_t block_count;
		size_t capacity;
	} MemoryTable;
\end{lstlisting}

The two main functions for this module is \texttt{read\textunderscore{}mem} and \texttt{write\textunderscore{}mem}. 
%
The former takes in as input a memory and returns the value at that address.
%
It first checks the L1 cache, then the L2 cache, then finally the RAM.
%
It also updates the hit/miss stats for the different caches.
%
The latter takes a memory address and a value and writes that value at the given memory address.
%
These are the core operations used to fetch and store information from the CPU to the main memory and vice versa.
\begin{lstlisting}[language=MyC, escapechar=\$, numbers=none]
	//return the value at the given memory address
	word read_mem(const mem_addr addr)
	{
		int index;
		index = cache_search(&L1, addr);
		if(index != EMPTY_ADDR)
		{
			//Cache hit at L1
			L1cache_hit++;
			return L1.items[index].val;
		}
		
		//cache miss at L1
		L1cache_miss++;
		
		index = cache_search(&L2, addr);
		if(index != EMPTY_ADDR)
		{
			//Cache hit at l2
			L2cache_hit++;
			word val = L2.items[index].val;
			//Update L1 cache to prevent future cache misses
			update_cache(&L1, addr, val);
			return val;
		}
		
		//Cache miss at L2
		L2cache_miss++;
		
		//Complete cache miss, so read RAM and update cache
		word val = RAM[addr];
		update_cache(&L1, addr, val);
		update_cache(&L2, addr, val);
		return val;
	}
	
	//write the given value to the given memory address.
	void write_mem(const mem_addr addr, const word val)
	{
		RAM[addr] = val;
		
		int index;
		
		//Update L1 Cache
		index = cache_search(&L1, addr);
		if(index != EMPTY_ADDR)
		{
			L1.items[index].val = val;
		}
		
		//Update L2 Cache
		index = cache_search(&L2, addr);
		if(index != EMPTY_ADDR)
		{
			L2.items[index].val = val;
		}
	}
\end{lstlisting}
%
The helper functions \texttt{cache\textunderscore{}search} and \texttt{update\textunderscore{}cache} are used to manage interfacing with the caches.
%
The former checks if the data can be found in the cache and if so returns the index.
%
The latter inserts data into the cache following the interface of a double ended queue.
%
\begin{lstlisting}[language=MyC, escapechar=\$, numbers=none]
	//find the address of the value if it exists in cache
	int cache_search(Cache* cache, const mem_addr addr)
	{
		for(int i = 0; i < cache->size; i++)
		{
			//the address we want was found in cache
			//so return the index of that address
			if(cache->items[i].addr == addr)
			{
				return i;
			}
		}
		//address not found so return signifier
		return EMPTY_ADDR;
	}
	
	//Update the given cache in case of misses
	void update_cache(Cache* cache, const mem_addr addr, const word val)
	{
		//calculate where in the cache to store the value
		int index = (cache->front + cache->count) % cache->size;
		cache->items[index].addr = addr;
		cache->items[index].val = val;
		
		//update the size and count of the cache
		if(cache->count < cache->size)
		{
			//cache isn't full so we can just put the new
			//value in the next index in the cache
			cache->count++;
		}
		else
		{
			//cache is full, so loop around to put the new
			//value at the front
			cache->front = (cache->front + 1) % cache->size;
		}
	}
\end{lstlisting}

\subsubsection{Memory Table}
We implement our Memory Table as a structure containing an array of memory blocks and a count.
%
Each memory block contains the ID of the process it belongs to, the starting address in memory, the ending address in memory, and a flag for if that block is free.
\begin{lstlisting}[language=MyC, escapechar=\$, numbers=none]
	typedef struct {
		int pid;
		dword start_addr;
		dword end_addr;
		bool is_free;
	} MemoryBlock;
	
	typedef struct
	{
		MemoryBlock *blocks;
		int block_count;
	} MemoryTable;
	
	void init_memtable(const int size)
	{
		//make the first entry into the table one large free block of memory
		MEMORY_TABLE.blocks = (MemoryBlock*)malloc(sizeof(MemoryBlock) * size);
		MEMORY_TABLE.blocks[0].pid = NO_PID;
		MEMORY_TABLE.blocks[0].is_free = true;
		MEMORY_TABLE.blocks[0].start_addr = 0;
		MEMORY_TABLE.blocks[0].end_addr = (size - 1);
		
		MEMORY_TABLE.block_count = 1;
		printf("initialized memory table with size %d \n" , size);
	}
\end{lstlisting}
\subsubsection{Dynamic Memory Allocation and Deallocation}
We use the best fit method to dynamically allocate memory.
\begin{lstlisting}[language=MyC, escapechar=\$, numbers=none]
	// allocate memory for a specific process
	// using the best fit method
	dword mallocate(int pid, int size)
	{
		int best_size = WRS + 1;
		int index = -1;
		
		for(int i = 0; i < MEMORY_TABLE.block_count; i++)
		{
			if (MEMORY_TABLE.blocks[i].is_free)
			{
				int mem_block_size = (MEMORY_TABLE.blocks[i].end_addr - MEMORY_TABLE.blocks[i].start_addr) + 1;
				if(mem_block_size >= size && mem_block_size < best_size)
				{
					best_size = mem_block_size;
					index = i;
				}
			}
		}
		
		//No memory free so ...idk
		if(index == -1)
		{
			printf("Could not fullfill process(PID %d)'s request for a (%d byte) chunk of memory: Not Enough Free Space\n", pid, size);
			return -1;
		}
		
		//Modify the free space to house our process
		//@david Boo! spooky mutation ~~ooooh~~
		MemoryBlock* best_fit = &MEMORY_TABLE.blocks[index];
		
		//save the old start and end addresses
		dword old_start_addr = best_fit->start_addr;
		dword old_end_addr = best_fit->end_addr;
		
		dword new_end_addr = (old_start_addr + size) - 1;
		
		//give the process the space
		best_fit->pid = pid;
		best_fit->is_free = false;
		best_fit->end_addr = new_end_addr;
		
		//cut down the size of the block
		//to free up unused space
		if(new_end_addr < old_end_addr)
		{
			//shift all the blocks to the right to make room
			for(int i = MEMORY_TABLE.block_count; i > index + 1; i--)
			{
				MEMORY_TABLE.blocks[i] = MEMORY_TABLE.blocks[i - 1];
			}
			
			MEMORY_TABLE.blocks[index +1].pid = NO_PID;
			MEMORY_TABLE.blocks[index +1].is_free = true;
			MEMORY_TABLE.blocks[index +1].start_addr = new_end_addr + 1;
			MEMORY_TABLE.blocks[index +1].end_addr = old_end_addr;
			
			MEMORY_TABLE.block_count++;
		}
		
		printf("Process (PID %d) given (%d bytes) of memory from [%d --> %d]\n", pid, size, best_fit->start_addr, best_fit->end_addr);
		return best_fit->start_addr;
	}
	
	// free up the memory block associated with the process
	void liberate(int pid)
	{
		int index = 0;
		for(index = 0; index < MEMORY_TABLE.block_count; index++)
		{
			if(MEMORY_TABLE.blocks[index].pid == pid && !MEMORY_TABLE.blocks[index].is_free)
			{
				MEMORY_TABLE.blocks[index].pid = NO_PID;
				MEMORY_TABLE.blocks[index].is_free = true;
				printf("Freed (PID %d) at memory [%d --> %d]\n", pid, MEMORY_TABLE.blocks[index].start_addr, MEMORY_TABLE.blocks[index].end_addr);
				break;
			}
		}
		//if the pid was not found
		if(index == MEMORY_TABLE.block_count){return;}
		
		//merge newly freed block with the previous block if it's also free
		if(index > 0 && MEMORY_TABLE.blocks[index - 1].is_free)
		{
			MEMORY_TABLE.blocks[index -1].end_addr = MEMORY_TABLE.blocks[index].end_addr;
			//shift everything left to clean the gap
			for(int i = index; i < MEMORY_TABLE.block_count - 1; i++)
			{
				MEMORY_TABLE.blocks[i] = MEMORY_TABLE.blocks[i + 1];
			}
			MEMORY_TABLE.block_count--;
			index--;
		}
		
		//merge with the next memory block if it's also free
		if(index < MEMORY_TABLE.block_count - 1 && MEMORY_TABLE.blocks[index + 1].is_free)
		{
			MEMORY_TABLE.blocks[index].end_addr = MEMORY_TABLE.blocks[index + 1].end_addr;
			//shift right to clean the gap
			for(int i = index + 1; i < MEMORY_TABLE.block_count - 1; i++)
			{
				MEMORY_TABLE.blocks[i] = MEMORY_TABLE.blocks[i + 1];
			}
			MEMORY_TABLE.block_count--;
	} }
\end{lstlisting}