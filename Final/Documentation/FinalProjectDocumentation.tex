\documentclass[titlepage, 11pt,a4paper]{article}

\usepackage[a4paper,margin=0.5in]{geometry}
\usepackage{amssymb}
\usepackage{booktabs}
\usepackage{array}
\usepackage{longtable}
\usepackage{subcaption}
\usepackage{amsmath}
\usepackage{hyperref}
\usepackage{cleveref}
\usepackage{alltt}
\usepackage{tikz}
\usepackage{float}
\floatstyle{ruled}
\restylefloat{figure}
\usepackage{pgfplots}
\pgfplotsset{compat=1.18}
\usepackage{balance}
\usepackage{listings}
\usepackage[utf8]{inputenc}
\usepackage{amsfonts}
\usepackage{graphicx}
\usepackage{stfloats}
\pgfplotsset{compat=newest}
\usepackage{pgfplotstable}
\usepackage{url}
\def\UrlBreaks{\do\/\do-}
\usepackage{bold-extra}
\usepackage{enumitem}
\setlist[description]{leftmargin=\parindent,labelindent=\parindent}
\usepackage{doi}
\usepackage{bm}
\usepackage[dvipsnames]{xcolor}

%\usepackage[outputdir]{lstlisting}
%\usepackage{fontspec}
%\setmonofont{JetBrainsMono Nerd Font}[
%    Scale=MatchLowercase
%]

\definecolor{pakistangreen}{rgb}{0.0, 0.4, 0.0}
\definecolor{palecerulean}{rgb}{0.61, 0.77, 0.89}
\definecolor{moonstoneblue}{rgb}{0.45, 0.66, 0.76}
\definecolor{oceanboatblue}{rgb}{0.0, 0.47, 0.75}
\definecolor{neonfuchsia}{rgb}{1.0, 0.25, 0.39}
\definecolor{palatinateblue}{rgb}{0.15, 0.23, 0.89}


\definecolor{vscKeyword}{RGB}{86,156,214}
\definecolor{vscType}{RGB}{78,201,176}
\definecolor{vscPreproc}{RGB}{197,134,192}
\definecolor{vscString}{RGB}{206,145,120}
\definecolor{vscComment}{RGB}{106,153,85}
\definecolor{vscFunction}{RGB}{220,220,180}
\definecolor{vscNumber}{RGB}{181,206,168}

\newcommand{\ADD}{\texttt{ADD}}
\newcommand{\SUB}{\texttt{SUB}}
\newcommand{\LOAD}{\texttt{LOAD}}
\newcommand{\STORE}{\texttt{STORE}}
\newcommand{\HALT}{\texttt{HALT}}
\newcommand{\PC}{\texttt{PC}}
\newcommand{\IR}{\texttt{IR}}
\newcommand{\ACC}{\texttt{ACC}}
\newcommand{\C}{\texttt{C}}
\newcommand{\N}{\texttt{n}}
\newcommand{\os}{\texttt{OS}}
\newcommand{\Update}{\textcolor{red}{\large{needs Updating!!}}}


\begin{document}
	
%defualt style for C listings
\lstset{
	% --- Linenos ---
	numbersep=3pt,                  % how far the line-numbers are from the code
	numberstyle=\tiny\color{gray},
	numbers=left,                   % where to put the line-numbers
	stepnumber=1,                   % the step between two line-numbers. If it is 1 each line will be numbered
	% -- Basics --
	basicstyle=\ttfamily\small,
	sensitive=true,  % Case-sensitive keywords
	tabsize=2,
	breaklines=true,  % Break lines if too long
	columns=fullflexible,
	keepspaces,
	showstringspaces=false,  % Spaces not shown as _
	upquote=true,
	%keywordstyle={\color{purple}}
}

\lstset{numbers=left,xleftmargin=2em,framexleftmargin=1.5em}

% Usage: \begin{lstlisting}[language=C] ...
	\lstdefinelanguage{MyC} {
		% -- Comments --
		morecomment=[l]{//},
		morecomment=[s]{/*}{*/},
		%morecomment=[l]{;},         % Inline comments start with ;
		%morecomment=[s]{\#|}{|\#},  % Block comments are done with #|  |#
		commentstyle={\color{pakistangreen}\slshape\sffamily},
		% -- Strings --
		morestring=[b]",
		stringstyle=\color{vscString},
		% --- Literal replacements ---
		literate=
		{\$}{\$}1,
		%{*}{{\color{cyan}{*}}}1
		%{&}{\color{blue}{&}}1
		%{&&}{{\color{cyan}{&&}}}1,
		%{||}{{\color{cyan}{||}}}1,
		%	{lambda}{{\$\lambda\$}}1
		%{->}{{\$\rightarrow\$}}1
		%{*}{{*}}1
		%{lambda}{{\textcolor{blue}{\(\lambda\)}}}1
		%{EMP}{{\ep{}}}1,
		%{'}{{\quot{}}}1,
		classoffset=1,
		morekeywords={
			void, main, typedef, struct, enum, static, int, float, double, bool, true, false, stderr, NULL, sizeof, const, char, unsigned, long, inline, printf, fprintf
		},
		keywordstyle=\color{oceanboatblue},
		classoffset=2,
		morekeywords={
			 if, for, while, else, switch, return, break, case, default 
		},
		keywordstyle=\color{neonfuchsia},
		classoffset=3,
		morekeywords={
			import, export
			},
		keywordstyle=\color{yellow},
		classoffset=4,
		morekeywords={
			Cpu, ProcessState, uint32_t, size_t, Process, Queue, MemoryBlock, Cache, CachePolicy, uint8_t, CacheLine, MemoryTable, uint16_t, int32_t, uint64_t, int64_t, AssemblyContext, int8_t, int16_t, DataSegment, TextEntry, Macro, Symbol, AssemblyResult, AssembledProgram, SymbolInfo, PerformanceTracker, ProcessMetrics, PerformanceMetrics, timespec, PerfTimer, SchedulingAlgorithm, IRQ, InterruptHeap, Interrupt, stack, Entry, dword, word, hword
			},
		keywordstyle=\color{palatinateblue},
		classoffset=5,
		morekeywords={
			define
		},
		keywordstyle=\color{vscPreproc},
		%classoffset=0,
		alsoletter={',`,-,/,>,<,\#,\$,?,=,&&},
		%moredelim=**[is][\color{lightgray}]{<<@<<}{>>@>>},
		%moredelim=**[is][\itshape\color{OliveGreen}]{<<;<<}{>>;>>},
	}
	% End of C listing style 

\begin{titlepage}
  \centering
  \vspace*{3cm}
  {\Huge\bfseries Final Project Documentation: \\ Advanced Operating System Simulator \\}
  \vspace*{2cm}
  {\Large\texttt{Brysen Pfingsten, Nathaniel Savoury, \\ David Fields}}
  \vfill
  {\large \today\par}
  \vspace{1cm}
  {\large CSAS 3111 - Operating Systems \\ Fall 2025 \\ Seton Hall University}
  \vspace*{2cm}
\end{titlepage}

\tableofcontents

\pagebreak

\section{Introduction}

\subsection{Problem Statement}
\label{problem-statement}

The goal of this project is to guide you through the process of building an advanced operating system simulator. It will challenge you to apply and integrate the key concepts you have learned in previous projects while introducing more advanced features such as multitasking, inter-process communication, memory hierarchy, and efficient process scheduling with real-time constraints.

The purpose of this project is to help you understand how a modern operating system manages CPU resources and handles tasks concurrently. You will explore various scheduling algorithms to ensure safe and efficient access to shared resources, all while simulating real-world CPU and memory behaviors. By the end of this project, you will have a solid understanding of how operating systems work and gain hands-on experience in building a fully functional OS simulator capable of efficiently managing multiple processes in a concurrent environment.

Eventually, this project aims to deepen your understanding of system-level programming and OS concepts, preparing you for real-world applications and advanced studies in computer science.

\subsection{Outline}




\begin{itemize}
	\item Module 1: Process Simulation
	\item Module 2: Advanced Memory Management
	\item Module 3: Process Scheduling and Context Switching
	\item Module 4: Interrupt Handling and Dispatcher
	\item Module 5: Efficiency Analysis of Concurrency
\end{itemize}


%\section{Key Concepts and Features}
%
%\subsection{Project 1}
%
%\subsection{Project 2}
%
%\subsection{Project 3}
%Wasn't Assigned
%
%\subsection{Project 4}
%Wasn't Assigned

\pagebreak


%\documentclass[11pt]{article}

%\usepackage[a4paper,margin=0.5in]{geometry}
%\usepackage{amsmath,amssymb}
%\usepackage{booktabs}
%\usepackage{array}
%\usepackage{hyperref}
%\usepackage{longtable}
%\usepackage{enumitem}

%\title{MIPS-I Subset 32-bit ISA Specification}
%\author{Operating Systems Final}
%\date{\today}

%\begin{document}
%\maketitle

%\tableofcontents

%\pagebreak
%\subsubsection{Core CPU Components and Registers}

\section{Overview}
\label{ISA-Spec}
This document specifies a small, real 32-bit ISA based closely on the original MIPS~I architecture. 
%
All instructions are 32 bits wide and follow one of three formats: R-type, I-type, or J-type.
%
The ISA includes:
%
\begin{itemize}[noitemsep]
  \item 32 general-purpose registers (GPRs).
  \item Special registers: \texttt{PC}, \texttt{HI}, \texttt{LO}.
  \item Basic system control registers: \texttt{Status}, \texttt{Cause}, \texttt{EPC}.
  \item Integer arithmetic and logical operations.
  \item Multiply and divide.
  \item Shift operations.
  \item Load and store instructions for bytes, halfwords, and words.
  \item Branch and jump instructions.
  \item System call and exception/interrupt return.
\end{itemize}

\section{Registers}

\subsection{General-Purpose Registers}

There are 32 general-purpose registers, each 32 bits wide.

\begin{center}
\begin{tabular}{@{}lll@{}}
\toprule
Number & Name & Role / Convention \\ \midrule
\$0  & \texttt{zero} & Constant zero, reads as 0, writes ignored \\
\$1  & \texttt{at}   & Assembler temporary \\
\$2--\$3  & \texttt{v0--v1} & Function return values \\
\$4--\$7  & \texttt{a0--a3} & Function arguments \\
\$8--\$15 & \texttt{t0--t7} & Temporaries (caller-saved) \\
\$16--\$23 & \texttt{s0--s7} & Saved registers (callee-saved) \\
\$24--\$25 & \texttt{t8--t9} & Temporaries \\
\$26--\$27 & \texttt{k0--k1} & Reserved for kernel / OS use \\
\$28 & \texttt{gp} & Global pointer \\
\$29 & \texttt{sp} & Stack pointer \\
\$30 & \texttt{fp/s8} & Frame pointer or extra saved register \\
\$31 & \texttt{ra} & Return address for calls (\texttt{JAL}, \texttt{JALR}) \\ \bottomrule
\end{tabular}
\end{center}

\subsection{Special Registers}

\begin{itemize}[noitemsep]
  \item \textbf{PC} (Program Counter): 32-bit address of the current instruction.
  \item \textbf{HI}, \textbf{LO}: 32-bit registers used to hold results of
        multiply and divide.
  \item \textbf{Status}: System status register (holds interrupt enable and mode bits).
  \item \textbf{Cause}: Encodes reason for last exception or interrupt.
  \item \textbf{EPC} (Exception Program Counter): Holds the address to return to after
        an exception, used by \texttt{ERET}.
\end{itemize}

\section{Instruction Formats}

All instructions are 32 bits. Bit 31 is the most significant bit (MSB).

\subsection{R-Type Format}

\begin{center}
\begin{tabular}{@{}cccccc@{}}
\toprule
31--26 & 25--21 & 20--16 & 15--11 & 10--6 & 5--0 \\
\texttt{opcode} & \texttt{rs} & \texttt{rt} & \texttt{rd} & \texttt{shamt} & \texttt{funct} \\ \midrule
6 bits & 5 bits & 5 bits & 5 bits & 5 bits & 6 bits \\ \bottomrule
\end{tabular}
\end{center}

For all R-type instructions:
\[
\texttt{opcode} = 0.
\]

\subsection{I-Type Format}

\begin{center}
\begin{tabular}{@{}cccc@{}}
\toprule
31--26 & 25--21 & 20--16 & 15--0 \\
\texttt{opcode} & \texttt{rs} & \texttt{rt} & \texttt{immediate} \\ \midrule
6 bits & 5 bits & 5 bits & 16 bits \\ \bottomrule
\end{tabular}
\end{center}

The 16-bit immediate is sign-extended or zero-extended depending on the instruction.

\subsection{J-Type Format}

\begin{center}
\begin{tabular}{@{}cc@{}}
\toprule
31--26 & 25--0 \\
\texttt{opcode} & \texttt{target} \\ \midrule
6 bits & 26 bits \\ \bottomrule
\end{tabular}
\end{center}

The effective jump address is formed as:
\[
\text{PC}_{\text{next}} = \bigl(\text{PC}_{\text{current}}[31{:}28] \ll 28\bigr)
                         \;|\; (\texttt{target} \ll 2).
\]

\section{Instruction Encoding and Semantics}

This section lists the instructions in the ISA, along with their encoding and semantic meaning.
%
All arithmetic is on 32-bit two's complement integers unless otherwise stated.
%
For brevity, we use the following notation:
\begin{itemize}[noitemsep]
  \item \(\mathrm{GPR}[i]\): contents of general-purpose register \(i\).
  \item \(\mathrm{PC}\): program counter.
  \item \(\mathrm{HI}, \mathrm{LO}\): special multiply/divide registers.
  \item \(\mathrm{Mem}[a]\): memory access at byte address \(a\).
  \item \(\mathrm{sext}_n(x)\): sign extension of \(x\) from \(n\) bits to 32 bits.
  \item \(\mathrm{zext}_n(x)\): zero extension of \(x\) from \(n\) bits to 32 bits.
\end{itemize}

\subsection{Integer Arithmetic (R-Type)}

All of these have \texttt{opcode} = 0.

\begin{longtable}{@{}lllll@{}}
\toprule
Mnemonic & Format & Encoding & Description & Semantics \\ \midrule
\endhead
ADD  & R & \texttt{funct} = 0x20 &
  Add (signed) &
  \(\mathrm{GPR}[rd] = \mathrm{GPR}[rs] + \mathrm{GPR}[rt]\) \\[0.2em]
ADDU & R & \texttt{funct} = 0x21 &
  Add (unsigned) &
  Same as \texttt{ADD} but ignore signed overflow \\[0.2em]
SUB  & R & \texttt{funct} = 0x22 &
  Subtract (signed) &
  \(\mathrm{GPR}[rd] = \mathrm{GPR}[rs] - \mathrm{GPR}[rt]\) \\[0.2em]
SUBU & R & \texttt{funct} = 0x23 &
  Subtract (unsigned) &
  Same as \texttt{SUB} but ignore signed overflow \\[0.2em]
\bottomrule
\end{longtable}

\subsection{Multiply and Divide (R-Type)}

{
\scriptsize
\begin{longtable}{@{}lllll@{}}
\toprule
Mnemonic & Format & Encoding & Description & Semantics \\ \midrule
\endhead
MULT  & R & \texttt{funct} = 0x18 &
  Signed multiply &
  \(\{\mathrm{HI}, \mathrm{LO}\} = \mathrm{sext}_{64}\bigl(\mathrm{GPR}[rs]\bigr) \times \mathrm{sext}_{64}\bigl(\mathrm{GPR}[rt]\bigr)\) \\[0.2em]
MULTU & R & \texttt{funct} = 0x19 &
  Unsigned multiply &
  \(\{\mathrm{HI}, \mathrm{LO}\} = \mathrm{zext}_{64}\bigl(\mathrm{GPR}[rs]\bigr) \times \mathrm{zext}_{64}\bigl(\mathrm{GPR}[rt]\bigr)\) \\[0.2em]
DIV   & R & \texttt{funct} = 0x1A &
  Signed divide &
  \(\mathrm{LO} = \mathrm{GPR}[rs] / \mathrm{GPR}[rt]\), \(\mathrm{HI} = \mathrm{GPR}[rs] \bmod \mathrm{GPR}[rt]\) \\[0.2em]
DIVU  & R & \texttt{funct} = 0x1B &
  Unsigned divide &
  \(\mathrm{LO} = \mathrm{GPR}[rs]_{u} / \mathrm{GPR}[rt]_{u}\), \(\mathrm{HI} = \mathrm{GPR}[rs]_{u} \bmod \mathrm{GPR}[rt]_{u}\) \\[0.2em]
MFHI  & R & \texttt{funct} = 0x10, \texttt{rs} = \texttt{rt} = 0 &
  Move from HI &
  \(\mathrm{GPR}[rd] = \mathrm{HI}\) \\[0.2em]
MFLO  & R & \texttt{funct} = 0x12, \texttt{rs} = \texttt{rt} = 0 &
  Move from LO &
  \(\mathrm{GPR}[rd] = \mathrm{LO}\) \\[0.2em]
MTHI  & R & \texttt{funct} = 0x11, \texttt{rt} = \texttt{rd} = 0 &
  Move to HI &
  \(\mathrm{HI} = \mathrm{GPR}[rs]\) \\[0.2em]
MTLO  & R & \texttt{funct} = 0x13, \texttt{rt} = \texttt{rd} = 0 &
  Move to LO &
  \(\mathrm{LO} = \mathrm{GPR}[rs]\) \\[0.2em]
\bottomrule
\end{longtable}
}

\subsection{Logical and Bitwise (R-Type)}

\begin{longtable}{@{}lllll@{}}
\toprule
Mnemonic & Format & Encoding & Description & Semantics \\ \midrule
\endhead
AND & R & \texttt{funct} = 0x24 &
  Bitwise AND &
  \(\mathrm{GPR}[rd] = \mathrm{GPR}[rs] \land \mathrm{GPR}[rt]\) \\[0.2em]
OR  & R & \texttt{funct} = 0x25 &
  Bitwise OR &
  \(\mathrm{GPR}[rd] = \mathrm{GPR}[rs] \lor \mathrm{GPR}[rt]\) \\[0.2em]
XOR & R & \texttt{funct} = 0x26 &
  Bitwise XOR &
  \(\mathrm{GPR}[rd] = \mathrm{GPR}[rs] \oplus \mathrm{GPR}[rt]\) \\[0.2em]
NOR & R & \texttt{funct} = 0x27 &
  Bitwise NOR &
  \(\mathrm{GPR}[rd] = \neg(\mathrm{GPR}[rs] \lor \mathrm{GPR}[rt])\) \\[0.2em]
\bottomrule
\end{longtable}

\subsection{Shift Instructions (R-Type)}

{
\footnotesize
\begin{longtable}{@{}lllll@{}}
\toprule
Mnemonic & Format & Encoding & Description & Semantics \\ \midrule
\endhead
SLL  & R & \texttt{funct} = 0x00 &
  Shift left logical (immediate) &
  \(\mathrm{GPR}[rd] = \mathrm{GPR}[rt] \ll \texttt{shamt}\) \\[0.2em]
SRL  & R & \texttt{funct} = 0x02 &
  Shift right logical (immediate) &
  \(\mathrm{GPR}[rd] = \mathrm{GPR}[rt] \gg \texttt{shamt}\) (logical) \\[0.2em]
SRA  & R & \texttt{funct} = 0x03 &
  Shift right arithmetic (immediate) &
  Arithmetic right shift, preserving sign bit \\[0.2em]
SLLV & R & \texttt{funct} = 0x04 &
  Shift left logical (variable) &
  \(\mathrm{GPR}[rd] = \mathrm{GPR}[rt] \ll (\mathrm{GPR}[rs] \,\&\, 0x1F)\) \\[0.2em]
SRLV & R & \texttt{funct} = 0x06 &
  Shift right logical (variable) &
  \(\mathrm{GPR}[rd] = \mathrm{GPR}[rt] \gg (\mathrm{GPR}[rs] \,\&\, 0x1F)\) (logical) \\[0.2em]
SRAV & R & \texttt{funct} = 0x07 &
  Shift right arithmetic (variable) &
  Arithmetic right shift by low 5 bits of \(\mathrm{GPR}[rs]\) \\[0.2em]
\bottomrule
\end{longtable}
}

\subsection{Immediate Arithmetic and Logical (I-Type)}

\begin{longtable}{@{}lllll@{}}
\toprule
Mnemonic & Format & Opcode & Description & Semantics \\ \midrule
\endhead
ADDI  & I & 0x08 &
  Add immediate (signed) &
  \(\mathrm{GPR}[rt] = \mathrm{GPR}[rs] + \mathrm{sext}_{16}(\texttt{imm})\) \\[0.2em]
ADDIU & I & 0x09 &
  Add immediate (unsigned) &
  Same as \texttt{ADDI} but ignore signed overflow \\[0.2em]
ANDI  & I & 0x0C &
  And immediate &
  \(\mathrm{GPR}[rt] = \mathrm{GPR}[rs] \land \mathrm{zext}_{16}(\texttt{imm})\) \\[0.2em]
ORI   & I & 0x0D &
  Or immediate &
  \(\mathrm{GPR}[rt] = \mathrm{GPR}[rs] \lor \mathrm{zext}_{16}(\texttt{imm})\) \\[0.2em]
XORI  & I & 0x0E &
  Xor immediate &
  \(\mathrm{GPR}[rt] = \mathrm{GPR}[rs] \oplus \mathrm{zext}_{16}(\texttt{imm})\) \\[0.2em]
SLTI  & I & 0x0A &
  Set less than immediate (signed) &
  \(\mathrm{GPR}[rt] = (\mathrm{GPR}[rs] < \mathrm{sext}_{16}(\texttt{imm})) ? 1 : 0\) \\[0.2em]
SLTIU & I & 0x0B &
  Set less than immediate (unsigned) &
  Unsigned comparison version of \texttt{SLTI} \\[0.2em]
LUI   & I & 0x0F &
  Load upper immediate &
  \(\mathrm{GPR}[rt] = \texttt{imm} \ll 16\) \\[0.2em]
\bottomrule
\end{longtable}

\subsection{Load and Store (I-Type)}

Effective address:
\[
\text{EA} = \mathrm{GPR}[rs] + \mathrm{sext}_{16}(\texttt{imm}).
\]

\noindent
Memory is typically treated as byte-addressed, little-endian.

\begin{longtable}{@{}lllll@{}}
\toprule
Mnemonic & Format & Opcode & Description & Semantics \\ \midrule
\endhead
LW  & I & 0x23 &
  Load word &
  \(\mathrm{GPR}[rt] = \text{Mem32}[\text{EA}]\) \\[0.2em]
SW  & I & 0x2B &
  Store word &
  \(\text{Mem32}[\text{EA}] = \mathrm{GPR}[rt]\) \\[0.2em]
LB  & I & 0x20 &
  Load byte (signed) &
  \(\mathrm{GPR}[rt] = \mathrm{sext}_{8}(\text{Mem8}[\text{EA}])\) \\[0.2em]
LBU & I & 0x24 &
  Load byte (unsigned) &
  \(\mathrm{GPR}[rt] = \mathrm{zext}_{8}(\text{Mem8}[\text{EA}])\) \\[0.2em]
LH  & I & 0x21 &
  Load halfword (signed) &
  \(\mathrm{GPR}[rt] = \mathrm{sext}_{16}(\text{Mem16}[\text{EA}])\) \\[0.2em]
LHU & I & 0x25 &
  Load halfword (unsigned) &
  \(\mathrm{GPR}[rt] = \mathrm{zext}_{16}(\text{Mem16}[\text{EA}])\) \\[0.2em]
SB  & I & 0x28 &
  Store byte &
  \(\text{Mem8}[\text{EA}] = \mathrm{GPR}[rt] \,\&\, 0xFF\) \\[0.2em]
SH  & I & 0x29 &
  Store halfword &
  \(\text{Mem16}[\text{EA}] = \mathrm{GPR}[rt] \,\&\, 0xFFFF\) \\[0.2em]
\bottomrule
\end{longtable}

\subsection{Branches (I-Type)}

The branch target address is computed relative to the address of the instruction
\emph{following} the branch.
%
Let \(\text{PC}_{\text{next}}\) be the PC after fetching the branch (i.e., \(\text{PC}+4\)).
Then:
\[
\text{Target} = \text{PC}_{\text{next}} + \bigl(\mathrm{sext}_{16}(\texttt{imm}) \ll 2 \bigr).
\]

\begin{longtable}{@{}lllll@{}}
\toprule
Mnemonic & Format & Opcode & Description & Semantics \\ \midrule
\endhead
BEQ & I & 0x04 &
  Branch if equal &
  If \(\mathrm{GPR}[rs] = \mathrm{GPR}[rt]\), then \(\mathrm{PC} = \text{Target}\) \\[0.2em]
BNE & I & 0x05 &
  Branch if not equal &
  If \(\mathrm{GPR}[rs] \neq \mathrm{GPR}[rt]\), then \(\mathrm{PC} = \text{Target}\) \\[0.2em]
\bottomrule
\end{longtable}

\subsection{Jumps (J-Type and R-Type)}

\begin{longtable}{@{}lllll@{}}
\toprule
Mnemonic & Format & Opcode/Funct & Description & Semantics \\ \midrule
\endhead
J   & J & \texttt{opcode} = 0x02 &
  Jump &
  \(\mathrm{PC} = (\mathrm{PC}_{\text{current}}[31{:}28] \ll 28) \;|\; (\texttt{target} \ll 2)\) \\[0.2em]
JAL & J & \texttt{opcode} = 0x03 &
  Jump and link &
  \(\mathrm{GPR}[31] = \mathrm{PC}_{\text{next}};\) then same as \texttt{J} \\[0.2em]
JR  & R & \texttt{funct} = 0x08 &
  Jump register &
  \(\mathrm{PC} = \mathrm{GPR}[rs]\) \\[0.2em]
JALR & R & \texttt{funct} = 0x09 &
  Jump and link register &
  \(\mathrm{GPR}[rd] = \mathrm{PC}_{\text{next}};\; \mathrm{PC} = \mathrm{GPR}[rs]\) \\[0.2em]
\bottomrule
\end{longtable}

\subsection{System and Exception Instructions}

For system and exception-related instructions, we describe them in prose rather
than putting lists inside table cells (which can cause LaTeX errors).

\begin{description}[style=nextline]
  \item[SYSCALL] 
    Encoded as an R-type instruction with \texttt{opcode} = 0 and
    \texttt{funct} = 0x0C. When executed, this instruction triggers a software
    exception. The simulator should:
    \begin{enumerate}[noitemsep]
      \item Save the appropriate instruction address into \texttt{EPC}
            (either the address of the syscall or the next instruction,
            depending on your chosen convention).
      \item Set \texttt{Cause} to a code representing a system call exception.
      \item Update \texttt{Status} to indicate kernel mode and (optionally)
            disable further interrupts.
      \item Set \texttt{PC} to the configured exception vector address
            (e.g.\ \texttt{0x80000180}).
    \end{enumerate}

  \item[BREAK]
    Encoded as an R-type instruction with \texttt{opcode} = 0 and
    \texttt{funct} = 0x0D. When executed, this triggers a breakpoint exception,
    which is handled similarly to \texttt{SYSCALL}, but with a different
    \texttt{Cause} code to distinguish it (e.g.\ for debugging or traps).
\end{description}

\subsection{Exception Return (ERET)}

In real MIPS this is encoded as a coprocessor 0 instruction. For this ISA we
define:

\begin{itemize}[noitemsep]
  \item \texttt{opcode} = 0x10 (COP0),
  \item \texttt{rs} = 0x10,
  \item bits 5--0 (funct) = 0x18,
  \item all other fields zero.
\end{itemize}

Decoding is implemented as a special case: ``if opcode is 0x10 and \texttt{funct} = 0x18, execute \texttt{ERET}.''

\paragraph{Semantics.}
\begin{itemize}[noitemsep]
  \item \(\mathrm{PC} \leftarrow \texttt{EPC}\).
  \item Restore user/kernel mode and interrupt enable bits in \texttt{Status}
        as appropriate.
\end{itemize}

\section{Exception and Interrupt Model}

\subsection{Exception Types}

Typical exception causes include:
\begin{itemize}[noitemsep]
  \item System call (\texttt{SYSCALL}).
  \item Breakpoint (\texttt{BREAK}).
  \item Arithmetic overflow (e.g., \texttt{ADD} with overflow).
  \item Invalid instruction.
  \item Address error on load/store.
  \item External interrupt (e.g., timer, I/O).
\end{itemize}

The simulator sets \texttt{Cause} to an integer code representing one of these
reasons.

\subsection{Exception Entry}

On an exception or interrupt, the CPU performs:
\begin{enumerate}[noitemsep]
  \item Save the faulting instruction address or the following address into
        \texttt{EPC}.
  \item Set \texttt{Cause} to the appropriate exception code.
  \item Modify \texttt{Status} to:
    \begin{itemize}[noitemsep]
      \item switch to kernel mode,
      \item optionally disable further interrupts.
    \end{itemize}
  \item Set \texttt{PC} to a fixed exception vector address, e.g.\ \texttt{0x80000180}.
\end{enumerate}

The kernel's exception handler at that address can then inspect \texttt{Cause},
\texttt{EPC}, and general registers to decide what to do.

\subsection{Exception Return}

When the kernel is finished handling the exception or interrupt, it executes
\texttt{ERET}, which:
\begin{itemize}[noitemsep]
  \item restores \texttt{PC} from \texttt{EPC},
  \item restores user/kernel mode (and possibly interrupt enable) from
        \texttt{Status}.
\end{itemize}

%\end{document}




\section{Module 2: Advanced Memory Management}
\label{Memory-Spec}

\subsection{Problem Statement}
\label{PS-m2}

\subsection{Implementation}
\label{IMP-m2}
\subsubsection{Hierarchical Memory System}

\subsubsection{Memory Table}

\subsubsection{Dynamic Memory Allocation and Deallocation}


%%%\section{Module 3: Process Scheduling and Context Switching}
\section{Module 3: Process Scheduling and Context Switching}
\label{Process-Spec}

\subsection{Problem Statement}
\label{PS-m3}

alla

\subsection{Implementation}
\label{IMP-m3}

akda

\subsubsection{Process Control Block Enhancements}

lsok

\subsubsection{Scheduling Algorithms}

ls
\subsubsection*{Round-Robin}
\label{RR}

ls

\subsubsection*{Priority-Based Scheduling}
\label{PBS}

lsf

\subsubsection*{Shortest Time Remaining}
\label{SRT}

lsf

\subsubsection*{Highest Response Ratio Next}
\label{HRRN}
lsv

\subsubsection*{First Come First Serve}
\label{FCFS}


kfs
\subsubsection*{Shortest Process Next}
\label{SPN}

 skf

\subsubsection*{Feedback Scheduling}
\label{FS}

jsnf

\subsubsection{Context Switching}


snfj
\subsubsection{Integration with Fetch-Decode-Execute Cycle}

jnsf


\section{Module 4: Interrupt Handling and Dispatcher}
\label{Interrupts-Spec}

las


\subsection{Problem Statement}
\label{PS-m4}
\subsection{Implementation}
\label{IMP-m4}

\subsubsection{Types of Interruption}

\subsubsection{Interrupt Vector Table}

%\subsubsection*{Interrupt Handler}
%\subsubsection*{Dispatcher}
\subsubsection{Context Switching}


\section{Module 5: Efficiency Analysis of Concurrency}
\label{OS-Performance}



\subsection{Problem Statement}
\label{PS-m5}

\subsection{Implementation}
\label{IMP-m5}

\subsubsection{Performance Metrics Setup}


\subsubsection{Implementation of Time Tracking}


\subsubsection{Data Comparison}


\subsubsection{Performance Comparison}


\subsubsection{Visualization and Reporting}

\section{The Simulation}
\label{simulation}
This program serves as an integrated system test for an advanced operating system simulator, demonstrating the interaction between multiple subsystems including the CPU, memory hierarchy, interrupt controller, and process management. It initializes all system components and assembles three programs (processes) — one prints “Hello, World” using CPU interrupts, another performs repeated arithmetic calculations and memory operations, and a third that is a combination of both I/O and CPU bound operations. At the same time, separate timer and I/O interrupt threads generate asynchronous events to test interrupt handling and CPU responsiveness. After these processes finish, a demo CPU program is loaded into memory and executed to verify instruction execution, arithmetic logic, and memory storage. Finally, the program outputs memory contents, CPU register states, and cache statistics before freeing all resources. Overall, it tests the system’s ability to handle processes, interrupt-driven execution, and coordinated CPU-memory operations within a simulated environment.

\begin{lstlisting}[language=MyC, escapechar=\$, numbers=none]
	static void run_single_algorithm(SchedulingAlgorithm algo) {
		// Reinitialize queues for fresh run
		free_queues();
		init_queues();
		
		// Reset process storage
		reset_process_storage();
		
		// Recreate processes
		for (int i = 0; i < opts.program_count; i++) {
			if (!results[i].success) continue;
			
			uint32_t process_addr = makeProcess(
			i,
			results[i].program->entry_point,
			results[i].program->text_start,
			results[i].program->text_size,
			results[i].program->data_start,
			results[i].program->data_size,
			results[i].program->stack_ptr,
			opts.priorities[i],
			results[i].program->text_size // opts.burst_estimates[i]
			);
			
			if (process_addr == UINT32_MAX) {
				fprintf(stderr, "Failed to recreate process %d\n", i);
			}
		}
		
		// Run the scheduler
		scheduler(algo);
	}
	
	int main(int argc, char *argv[]) {
		int exit_code = EXIT_SUCCESS;
		
		signal(SIGSEGV, panic_handler);
		signal(SIGABRT, panic_handler);
		signal(SIGFPE, panic_handler);
		
		int panic_signal = setjmp(g_panic_buffer);
		if (panic_signal != 0) {
			fprintf(stderr, "Performing emergency cleanup after signal %d\n", panic_signal);
			exit_code = EXIT_FAILURE;
			goto cleanup;
		}
		
		parse_args(argc, argv);
		
		// In comparison mode, keep allocations around so the same
		// text/data segments can be reused across multiple runs.
		if (opts.compare_all_algorithms) {
			set_memory_freeze(true);
		}
		
		// Initialize memory system
		printf("Initializing memory system...\n");
		init_memory(opts.cache_policy);
		memory_initialized = true;
		
		// Initialize process queues
		printf("Initializing process queues...\n");
		init_queues();
		queues_initialized = true;
		
		// Initialize performance tracking
		printf("Initializing performance tracking...\n");
		init_performance_tracking();
		perf_initialized = true;
		
		// Allocate results array
		results = calloc(opts.program_count, sizeof(AssemblyResult));
		result_count = opts.program_count;
		if(!results){
			fprintf(stderr, "Memory allocation failed for assembly results\n");
			exit_code = EXIT_FAILURE;
			goto cleanup;
		}
		
		// Assemble all programs
		printf("\n=== Assembling Programs ===\n");
		for (int i = 0; i < opts.program_count; i++) {
			printf("\n[%d/%d] Processing: %s\n", i+1, opts.program_count, opts.program_files[i]);
			
			results[i] = assemble(opts.program_files[i], i);
			
			if (!results[i].success) {
				fprintf(stderr, "  Assembly failed: %s\n", results[i].error_message);
				continue;
			}
			
			printf("   Assembly successful\n");
		}
		
		if (opts.compare_all_algorithms) {
			// Run all algorithms for comparison
			printf("\n");
			printf("================================================================================\n");
			printf("           RUNNING ALL SCHEDULING ALGORITHMS FOR COMPARISON\n");
			printf("================================================================================\n");
			
			SchedulingAlgorithm algorithms[] = {
				SCHED_FCFS,
				SCHED_ROUND_ROBIN,
				SCHED_SPN,
				SCHED_SRT,
				SCHED_PRIORITY,
				SCHED_HRRN,
				SCHED_MLFQ
			};
			
			const char *algo_names[] = {
				"FCFS",
				"Round Robin",
				"SPN (Shortest Process Next)",
				"SRT (Shortest Remaining Time)",
				"Priority",
				"HRRN (Highest Response Ratio Next)",
				"MLFQ (Multi-Level Feedback Queue)"
			};
			
			int num_algorithms = sizeof(algorithms) / sizeof(algorithms[0]);
			
			for (int i = 0; i < num_algorithms; i++) {
				printf("\n");
				printf("********************************************************************************\n");
				printf("                    Running: %s\n", algo_names[i]);
				printf("********************************************************************************\n");
				
				run_single_algorithm(algorithms[i]);
				
				printf("\n");
			}
			
			// Print comparison after all algorithms
			printf("\n");
			printf("################################################################################\n");
			printf("#                        FINAL COMPARISON REPORT                               #\n");
			printf("################################################################################\n");
			print_comparison_table();
			
			if (opts.export_csv) {
				export_comparison_csv(opts.csv_filename);
				
				// Generate individual chart data files
				generate_chart_data("waiting_time", "chart_waiting_time.csv");
				generate_chart_data("cpu_util", "chart_cpu_utilization.csv");
				generate_chart_data("context_switches", "chart_context_switches.csv");
			}
			
		} else {
			// Run single algorithm
			printf("\n");
			printf("================================================================================\n");
			printf("                    Creating Processes for Scheduling\n");
			printf("================================================================================\n");
			
			for (int i = 0; i < opts.program_count; i++) {
				if (!results[i].success) continue;
				
				uint32_t process_addr = makeProcess(
				i,
				results[i].program->entry_point,
				results[i].program->text_start,
				results[i].program->text_size,
				results[i].program->data_start,
				results[i].program->data_size,
				results[i].program->stack_ptr,
				opts.priorities[i],
				opts.burst_estimates[i]
				);
				
				if (process_addr == UINT32_MAX) {
					fprintf(stderr, "   Failed to create process\n");
					exit_code = EXIT_FAILURE;
				} else {
					printf("   Process created (PID: %d, Entry: 0x%08x)\n", 
					i, results[i].program->entry_point);
				}
			}
			
			// Run the scheduler
			printf("\n=== Starting Scheduler ===\n");
			printf("Algorithm: ");
			switch (opts.scheduler) {
				case SCHED_FCFS: printf("First-Come First-Served\n"); break;
				case SCHED_ROUND_ROBIN: printf("Round Robin\n"); break;
				case SCHED_PRIORITY: printf("Priority\n"); break;
				case SCHED_SRT: printf("Shortest Remaining Time\n"); break;
				case SCHED_HRRN: printf("Highest Response Ratio Next\n"); break;
				case SCHED_SPN: printf("Shortest Process Next\n"); break;
				case SCHED_MLFQ: printf("Multi-Level Feedback Queue\n"); break;
			}
			printf("\n");
			
			scheduler(opts.scheduler);
		}
		
		printf("\n=== Execution Complete ===\n");
		print_cache_stats();
		
		cleanup:
		g_panic_handler_active = 1;
		
		if (results) {
			for (int i = 0; i < result_count; i++) {
				if (results[i].success) {
					free_program(&results[i]);
				}
			}
			free(results);
			results = NULL;
		}
		
		if (opts.program_files) {
			free((void*)opts.program_files);
			opts.program_files = NULL;
		}
		
		if (opts.priorities) {
			free(opts.priorities);
			opts.priorities = NULL;
		}
		
		if (opts.burst_estimates) {
			free(opts.burst_estimates);
			opts.burst_estimates = NULL;
		}
		
		if (perf_initialized) {
			free_performance_tracking();
			perf_initialized = false;
		}
		
		if (memory_initialized) {
			free_memory();
			memory_initialized = false;
		}
		
		if (queues_initialized){
			free_queues();
			queues_initialized = false;
		}
		
		g_panic_handler_active = 0;
		return exit_code;
	}
\end{lstlisting}

And the output
\begin{alltt}
	Initializing memory system...
	Initialized cache at -> '0x5c6a83b15e80' <- with | 64 | bytes, and | 1 | lines [Line Size = 64]
	Initialized cache at -> '0x5c6a83b15e40' <- with | 128 | bytes, and | 2 | lines [Line Size = 64]
	Memory initialized with write-through cache policy
	Initializing process queues...
	Initializing performance tracking...
	Performance tracking initialized
	
	=== Assembling Programs ===
	
	[1/3] Processing: programs/factorial.asm
	mallocate: PID 0 allocated 136 bytes [0 -> 135]
	Allocated text: 0x00000000 (136 bytes)
	mallocate: PID 0 allocated 22 bytes [136 -> 157]
	 Allocated data: 0x00000088 (22 bytes)
	mallocate: PID 0 allocated 4096 bytes [160 -> 4255]
	 Assembly successful
	
	[2/3] Processing: programs/goodbye\_planet.asm
	mallocate: PID 1 allocated 72 bytes [4256 -> 4327]
	Allocated text: 0x000010a0 (72 bytes)
	mallocate: PID 1 allocated 47 bytes [4328 -> 4374]
	 Allocated data: 0x000010e8 (47 bytes)
	mallocate: PID 1 allocated 4096 bytes [4376 -> 8471]
	 Assembly successful
	
	[3/3] Processing: programs/hello\_world.asm
	mallocate: PID 2 allocated 24 bytes [8472 -> 8495]
	Allocated text: 0x00002118 (24 bytes)
	mallocate: PID 2 allocated 15 bytes [8496 -> 8510]
	 Allocated data: 0x00002130 (15 bytes)
	mallocate: PID 2 allocated 4096 bytes [8512 -> 12607]
	 Assembly successful
	
	================================================================================
	RUNNING ALL SCHEDULING ALGORITHMS FOR COMPARISON
	================================================================================
	
	********************************************************************************
	Running: FCFS
	********************************************************************************
	 Process created:
	PID: 0
	PC:  0x00000000
	SP:  0x0000109c
	Text: 0x00000000 - 0x00000088 (136 bytes)
	Data: 0x00000088 - 0x0000009e (22 bytes)
	Priority: 1, Burst: 136
	 Process created:
	PID: 1
	PC:  0x000010a0
	SP:  0x00002114
	Text: 0x000010a0 - 0x000010e8 (72 bytes)
	Data: 0x000010e8 - 0x00001117 (47 bytes)
	Priority: 2, Burst: 72
	 Process created:
	PID: 2
	PC:  0x00002118
	SP:  0x0000313c
	Text: 0x00002118 - 0x00002130 (24 bytes)
	Data: 0x00002130 - 0x0000213f (15 bytes)
	Priority: 3, Burst: 24
	
	=== Starting performance tracking for: FCFS ===
	
	Scheduling algorithm: FCFS
	Total 3 tasks to be scheduled
	=============================
	<system time 0> process 0 starts running
	Factorial result: 120
	
	<system time 70> process 0 finished.
	<system time 70> process 1 starts running
	Countdown: Countdown: 321Goodbye, Planet Earth!
	<system time 101> process 1 finished.
	<system time 101> process 2 starts running
	Hello, World!
	<system time 106> process 2 finished.
	<system time 106> All processes finished.
	=== Performance tracking completed for: FCFS ===
	
	
	========================================================================
	ALGORITHM: FCFS
	========================================================================
	
	Timing Metrics:
	Total Execution Time:      106.000 time units
	CPU Active Time:           106.000 time units
	CPU Idle Time:             0.000 time units
	Scheduler Overhead:        0.357 ms
	Context Switch Overhead:   0.082 ms
	
	Process Metrics:
	Number of Processes:       3
	Average Waiting Time:      57.000
	Average Turnaround Time:   92.333
	Average Response Time:     57.000
	
	System Metrics:
	CPU Utilization:           100.00\%
	Throughput:                0.028 processes/unit
	Context Switches:          3
	
	Memory Statistics:
	L1 Cache Hits:             499
	L1 Cache Misses:           41
	L1 Hit Rate:               92.41\%
	L2 Cache Hits:             20
	L2 Cache Misses:           21
	L2 Hit Rate:               48.78\%
	Write-Backs:               0
	
	PROCESS  ARRIVAL  BURST  COMPLETION  WAITING  TURNAROUND  RESPONSE  PRIORITY
	===============================================================================
	P0       0        70     70          0        70          0         1
	P1       0        31     101         70       101         70        2
	P2       0        5      106         101      106         101       3
	===============================================================================
	
	
	
	********************************************************************************
	Running: Round Robin
	********************************************************************************
	 Process created:
	PID: 0
	PC:  0x00000000
	SP:  0x0000109c
	Text: 0x00000000 - 0x00000088 (136 bytes)
	Data: 0x00000088 - 0x0000009e (22 bytes)
	Priority: 1, Burst: 136
	 Process created:
	PID: 1
	PC:  0x000010a0
	SP:  0x00002114
	Text: 0x000010a0 - 0x000010e8 (72 bytes)
	Data: 0x000010e8 - 0x00001117 (47 bytes)
	Priority: 2, Burst: 72
	 Process created:
	PID: 2
	PC:  0x00002118
	SP:  0x0000313c
	Text: 0x00002118 - 0x00002130 (24 bytes)
	Data: 0x00002130 - 0x0000213f (15 bytes)
	Priority: 3, Burst: 24
	
	=== Starting performance tracking for: Round Robin ===
	
	Scheduling algorithm: Round Robin
	Total 3 tasks to be scheduled
	=============================
	<system time 0> process 0 starts running
	<system time 3> process 1 starts running
	Countdown: Countdown: <system time 6> process 2 starts running
	Hello, World!
	<system time 9> process 0 starts running
	<system time 12> process 1 starts running
	<system time 15> process 2 starts running
	<system time 17> process 2 finished.
	<system time 17> process 0 starts running
	<system time 20> process 1 starts running
	3<system time 23> process 0 starts running
	<system time 26> process 1 starts running
	<system time 29> process 0 starts running
	<system time 32> process 1 starts running
	<system time 35> process 0 starts running
	<system time 38> process 1 starts running
	2<system time 41> process 0 starts running
	<system time 44> process 1 starts running
	<system time 47> process 0 starts running
	<system time 50> process 1 starts running
	1<system time 53> process 0 starts running
	<system time 56> process 1 starts running
	<system time 59> process 0 starts running
	<system time 62> process 1 starts running
	Goodbye, Planet Earth!
	<system time 65> process 0 starts running
	<system time 68> process 1 starts running
	<system time 69> process 1 finished.
	<system time 69> process 0 starts running
	<system time 72> process 0 starts running
	<system time 75> process 0 starts running
	<system time 78> process 0 starts running
	<system time 81> process 0 starts running
	<system time 84> process 0 starts running
	<system time 87> process 0 starts running
	<system time 90> process 0 starts running
	<system time 93> process 0 starts running
	<system time 96> process 0 starts running
	Factorial result: <system time 99> process 0 starts running
	120<system time 102> process 0 starts running
	
	
	<system time 105> process 0 starts running
	<system time 106> process 0 finished.
	<system time 106> All processes finished.
	=== Performance tracking completed for: Round Robin ===
	
	
	========================================================================
	ALGORITHM: Round Robin
	========================================================================
	
	Timing Metrics:
	Total Execution Time:      106.000 time units
	CPU Active Time:           106.000 time units
	CPU Idle Time:             0.000 time units
	Scheduler Overhead:        0.504 ms
	Context Switch Overhead:   0.065 ms
	
	Process Metrics:
	Number of Processes:       3
	Average Waiting Time:      28.667
	Average Turnaround Time:   64.000
	Average Response Time:     3.000
	
	System Metrics:
	CPU Utilization:           100.00\%
	Throughput:                0.028 processes/unit
	Context Switches:          37
	
	Memory Statistics:
	L1 Cache Hits:             979
	L1 Cache Misses:           101
	L1 Hit Rate:               90.65\%
	L2 Cache Hits:             45
	L2 Cache Misses:           56
	L2 Hit Rate:               44.55\%
	Write-Backs:               0
	
	PROCESS  ARRIVAL  BURST  COMPLETION  WAITING  TURNAROUND  RESPONSE  PRIORITY
	===============================================================================
	P2       0        5      17          12       17          6         3
	P1       0        31     69          38       69          3         2
	P0       0        70     106         36       106         0         1
	===============================================================================
	
	
	
	********************************************************************************
	Running: SPN (Shortest Process Next)
	********************************************************************************
	 Process created:
	PID: 0
	PC:  0x00000000
	SP:  0x0000109c
	Text: 0x00000000 - 0x00000088 (136 bytes)
	Data: 0x00000088 - 0x0000009e (22 bytes)
	Priority: 1, Burst: 136
	 Process created:
	PID: 1
	PC:  0x000010a0
	SP:  0x00002114
	Text: 0x000010a0 - 0x000010e8 (72 bytes)
	Data: 0x000010e8 - 0x00001117 (47 bytes)
	Priority: 2, Burst: 72
	 Process created:
	PID: 2
	PC:  0x00002118
	SP:  0x0000313c
	Text: 0x00002118 - 0x00002130 (24 bytes)
	Data: 0x00002130 - 0x0000213f (15 bytes)
	Priority: 3, Burst: 24
	
	=== Starting performance tracking for: SPN ===
	
	Scheduling algorithm: SPN (Shortest Process Next)
	Total 3 tasks to be scheduled
	=============================
	<system time 0> process 2 starts running
	Hello, World!
	<system time 5> process 2 finished.
	<system time 5> process 0 starts running
	Factorial result: 120
	
	<system time 75> process 0 finished.
	<system time 75> process 1 starts running
	Countdown: Countdown: 321Goodbye, Planet Earth!
	<system time 106> process 1 finished.
	<system time 106> All processes finished.
	=== Performance tracking completed for: SPN ===
	
	
	========================================================================
	ALGORITHM: SPN
	========================================================================
	
	Timing Metrics:
	Total Execution Time:      106.000 time units
	CPU Active Time:           106.000 time units
	CPU Idle Time:             0.000 time units
	Scheduler Overhead:        0.208 ms
	Context Switch Overhead:   0.159 ms
	
	Process Metrics:
	Number of Processes:       3
	Average Waiting Time:      26.667
	Average Turnaround Time:   62.000
	Average Response Time:     26.667
	
	System Metrics:
	CPU Utilization:           100.00\%
	Throughput:                0.028 processes/unit
	Context Switches:          3
	
	Memory Statistics:
	L1 Cache Hits:             1478
	L1 Cache Misses:           142
	L1 Hit Rate:               91.23\%
	L2 Cache Hits:             66
	L2 Cache Misses:           76
	L2 Hit Rate:               46.48\%
	Write-Backs:               0
	
	PROCESS  ARRIVAL  BURST  COMPLETION  WAITING  TURNAROUND  RESPONSE  PRIORITY
	===============================================================================
	P2       0        5      5           0        5           0         3
	P0       0        70     75          5        75          5         1
	P1       0        31     106         75       106         75        2
	===============================================================================
	
	
	
	********************************************************************************
	Running: SRT (Shortest Remaining Time)
	********************************************************************************
	 Process created:
	PID: 0
	PC:  0x00000000
	SP:  0x0000109c
	Text: 0x00000000 - 0x00000088 (136 bytes)
	Data: 0x00000088 - 0x0000009e (22 bytes)
	Priority: 1, Burst: 136
	 Process created:
	PID: 1
	PC:  0x000010a0
	SP:  0x00002114
	Text: 0x000010a0 - 0x000010e8 (72 bytes)
	Data: 0x000010e8 - 0x00001117 (47 bytes)
	Priority: 2, Burst: 72
	 Process created:
	PID: 2
	PC:  0x00002118
	SP:  0x0000313c
	Text: 0x00002118 - 0x00002130 (24 bytes)
	Data: 0x00002130 - 0x0000213f (15 bytes)
	Priority: 3, Burst: 24
	
	=== Starting performance tracking for: SRT ===
	
	Scheduling algorithm: SRT (Shortest Remaining Time)
	Total 3 tasks to be scheduled
	=============================
	Hello, World!
	<system time 5> process 2 finished.
	Countdown: Countdown: 321Goodbye, Planet Earth!
	<system time 37> process 1 finished.
	Factorial result: 120
	
	<system time 106> process 0 finished.
	<system time 106> All processes finished.
	=== Performance tracking completed for: SRT ===
	
	
	========================================================================
	ALGORITHM: SRT
	========================================================================
	
	Timing Metrics:
	Total Execution Time:      106.000 time units
	CPU Active Time:           106.000 time units
	CPU Idle Time:             0.000 time units
	Scheduler Overhead:        0.149 ms
	Context Switch Overhead:   0.001 ms
	
	Process Metrics:
	Number of Processes:       3
	Average Waiting Time:      14.000
	Average Turnaround Time:   49.333
	Average Response Time:     3.667
	
	System Metrics:
	CPU Utilization:           100.00\%
	Throughput:                0.028 processes/unit
	Context Switches:          4
	
	Memory Statistics:
	L1 Cache Hits:             1976
	L1 Cache Misses:           184
	L1 Hit Rate:               91.48\%
	L2 Cache Hits:             86
	L2 Cache Misses:           98
	L2 Hit Rate:               46.74\%
	Write-Backs:               0
	
	PROCESS  ARRIVAL  BURST  COMPLETION  WAITING  TURNAROUND  RESPONSE  PRIORITY
	===============================================================================
	P2       0        5      5           0        5           0         3
	P1       0        31     37          6        37          6         2
	P0       0        70     106         36       106         5         1
	===============================================================================
	
	
	
	********************************************************************************
	Running: Priority
	********************************************************************************
	 Process created:
	PID: 0
	PC:  0x00000000
	SP:  0x0000109c
	Text: 0x00000000 - 0x00000088 (136 bytes)
	Data: 0x00000088 - 0x0000009e (22 bytes)
	Priority: 1, Burst: 136
	 Process created:
	PID: 1
	PC:  0x000010a0
	SP:  0x00002114
	Text: 0x000010a0 - 0x000010e8 (72 bytes)
	Data: 0x000010e8 - 0x00001117 (47 bytes)
	Priority: 2, Burst: 72
	 Process created:
	PID: 2
	PC:  0x00002118
	SP:  0x0000313c
	Text: 0x00002118 - 0x00002130 (24 bytes)
	Data: 0x00002130 - 0x0000213f (15 bytes)
	Priority: 3, Burst: 24
	
	=== Starting performance tracking for: Priority ===
	
	Scheduling algorithm: Priority
	Total 3 tasks to be scheduled
	=============================
	Factorial result: 120
	
	<system time 70> process 0 finished.
	Countdown: Countdown: 321Goodbye, Planet Earth!
	<system time 101> process 1 finished.
	Hello, World!
	<system time 106> process 2 finished.
	<system time 106> All processes finished.
	=== Performance tracking completed for: Priority ===
	
	
	========================================================================
	ALGORITHM: Priority
	========================================================================
	
	Timing Metrics:
	Total Execution Time:      106.000 time units
	CPU Active Time:           106.000 time units
	CPU Idle Time:             0.000 time units
	Scheduler Overhead:        0.235 ms
	Context Switch Overhead:   0.000 ms
	
	Process Metrics:
	Number of Processes:       3
	Average Waiting Time:      57.000
	Average Turnaround Time:   92.333
	Average Response Time:     57.000
	
	System Metrics:
	CPU Utilization:           100.00\%
	Throughput:                0.028 processes/unit
	Context Switches:          3
	
	Memory Statistics:
	L1 Cache Hits:             2476
	L1 Cache Misses:           224
	L1 Hit Rate:               91.70\%
	L2 Cache Hits:             106
	L2 Cache Misses:           118
	L2 Hit Rate:               47.32\%
	Write-Backs:               0
	
	PROCESS  ARRIVAL  BURST  COMPLETION  WAITING  TURNAROUND  RESPONSE  PRIORITY
	===============================================================================
	P0       0        70     70          0        70          0         1
	P1       0        31     101         70       101         70        2
	P2       0        5      106         101      106         101       3
	===============================================================================
	
	
	
	********************************************************************************
	Running: HRRN (Highest Response Ratio Next)
	********************************************************************************
	 Process created:
	PID: 0
	PC:  0x00000000
	SP:  0x0000109c
	Text: 0x00000000 - 0x00000088 (136 bytes)
	Data: 0x00000088 - 0x0000009e (22 bytes)
	Priority: 1, Burst: 136
	 Process created:
	PID: 1
	PC:  0x000010a0
	SP:  0x00002114
	Text: 0x000010a0 - 0x000010e8 (72 bytes)
	Data: 0x000010e8 - 0x00001117 (47 bytes)
	Priority: 2, Burst: 72
	 Process created:
	PID: 2
	PC:  0x00002118
	SP:  0x0000313c
	Text: 0x00002118 - 0x00002130 (24 bytes)
	Data: 0x00002130 - 0x0000213f (15 bytes)
	Priority: 3, Burst: 24
	
	=== Starting performance tracking for: HRRN ===
	
	Scheduling algorithm: HRRN (Highest Response Ratio Next)
	Total 3 tasks to be scheduled
	=============================
	<system time 0> process 0 starts running
	Factorial result: 120
	
	<system time 70> process 0 finished.
	<system time 70> process 2 starts running
	Hello, World!
	<system time 75> process 2 finished.
	<system time 75> process 1 starts running
	Countdown: Countdown: 321Goodbye, Planet Earth!
	<system time 106> process 1 finished.
	<system time 106> All processes finished.
	=== Performance tracking completed for: HRRN ===
	
	
	========================================================================
	ALGORITHM: HRRN
	========================================================================
	
	Timing Metrics:
	Total Execution Time:      106.000 time units
	CPU Active Time:           106.000 time units
	CPU Idle Time:             0.000 time units
	Scheduler Overhead:        0.269 ms
	Context Switch Overhead:   0.200 ms
	
	Process Metrics:
	Number of Processes:       3
	Average Waiting Time:      48.333
	Average Turnaround Time:   83.667
	Average Response Time:     48.333
	
	System Metrics:
	CPU Utilization:           100.00\%
	Throughput:                0.028 processes/unit
	Context Switches:          3
	
	Memory Statistics:
	L1 Cache Hits:             2975
	L1 Cache Misses:           265
	L1 Hit Rate:               91.82\%
	L2 Cache Hits:             126
	L2 Cache Misses:           139
	L2 Hit Rate:               47.55\%
	Write-Backs:               0
	
	PROCESS  ARRIVAL  BURST  COMPLETION  WAITING  TURNAROUND  RESPONSE  PRIORITY
	===============================================================================
	P0       0        70     70          0        70          0         1
	P2       0        5      75          70       75          70        3
	P1       0        31     106         75       106         75        2
	===============================================================================
	
	
	
	********************************************************************************
	Running: MLFQ (Multi-Level Feedback Queue)
	********************************************************************************
	 Process created:
	PID: 0
	PC:  0x00000000
	SP:  0x0000109c
	Text: 0x00000000 - 0x00000088 (136 bytes)
	Data: 0x00000088 - 0x0000009e (22 bytes)
	Priority: 1, Burst: 136
	 Process created:
	PID: 1
	PC:  0x000010a0
	SP:  0x00002114
	Text: 0x000010a0 - 0x000010e8 (72 bytes)
	Data: 0x000010e8 - 0x00001117 (47 bytes)
	Priority: 2, Burst: 72
	 Process created:
	PID: 2
	PC:  0x00002118
	SP:  0x0000313c
	Text: 0x00002118 - 0x00002130 (24 bytes)
	Data: 0x00002130 - 0x0000213f (15 bytes)
	Priority: 3, Burst: 24
	
	=== Starting performance tracking for: MLFQ ===
	
	Scheduling algorithm: MLFQ (Multi-Level Feedback Queue)
	Total processes to be scheduled
	=============================
	Countdown: Countdown: Hello, World!
	<system time 17> process 2 finished.
	Factorial result: 120
	
	<system time 81> process 0 finished.
	321Goodbye, Planet Earth!
	<system time 106> process 1 finished.
	<system time 106> All processes finished.
	=== Performance tracking completed for: MLFQ ===
	
	
	========================================================================
	ALGORITHM: MLFQ
	========================================================================
	
	Timing Metrics:
	Total Execution Time:      106.000 time units
	CPU Active Time:           106.000 time units
	CPU Idle Time:             0.000 time units
	Scheduler Overhead:        0.397 ms
	Context Switch Overhead:   0.116 ms
	
	Process Metrics:
	Number of Processes:       3
	Average Waiting Time:      32.667
	Average Turnaround Time:   68.000
	Average Response Time:     2.000
	
	System Metrics:
	CPU Utilization:           100.00\%
	Throughput:                0.028 processes/unit
	Context Switches:          8
	
	Memory Statistics:
	L1 Cache Hits:             3470
	L1 Cache Misses:           310
	L1 Hit Rate:               91.80\%
	L2 Cache Hits:             148
	L2 Cache Misses:           162
	L2 Hit Rate:               47.74\%
	Write-Backs:               0
	
	PROCESS  ARRIVAL  BURST  COMPLETION  WAITING  TURNAROUND  RESPONSE  PRIORITY
	===============================================================================
	P2       0        5      17          12       17          4         3
	P0       0        70     81          11       81          0         1
	P1       0        31     106         75       106         2         2
	===============================================================================
	
	
	
	################################################################################
	#                        FINAL COMPARISON REPORT                               #
	################################################################################
	
	=====================================================================================================
	SCHEDULING ALGORITHM COMPARISON
	=====================================================================================================
	Algorithm         Avg Wait Avg T.Around     Avg Resp      CPU%%   C.Switches
	-----------------------------------------------------------------------------------------------------
	FCFS                57.000       92.333       57.000    100.00\%            3
	Round Robin         28.667       64.000        3.000    100.00\%           37
	SPN                 26.667       62.000       26.667    100.00\%            3
	SRT                 14.000       49.333        3.667    100.00\%            4
	Priority            57.000       92.333       57.000    100.00\%            3
	HRRN                48.333       83.667       48.333    100.00\%            3
	MLFQ                32.667       68.000        2.000    100.00\%            8
	=====================================================================================================
	
	Best Performers:
	Lowest Average Waiting Time:    SRT (14.000)
	Lowest Average Turnaround Time: SRT (49.333)
	Lowest Average Response Time:   MLFQ (2.000)
	Highest CPU Utilization:        FCFS (100.00\%)
	Fewest Context Switches:        FCFS (3)
	
	
	=== Execution Complete ===
	
	=== Cache Statistics ===
	L1 Cache:
	Hits:   3470
	Misses: 310
	Hit Rate: 91.80\%
	
	L2 Cache:
	Hits:   148
	Misses: 162
	Hit Rate: 47.74\%
	========================
\end{alltt}


\section{Testing and Debugging}
\label{testNdebug}
This test program evaluates the complete functionality and integration of the simulated computer system, including CPU execution, memory management, interrupt handling and process schedulers. It launches three concurrent processes: one that prints a message using CPU interrupts, another that performs intensive arithmetic operations and memory storage, and a third that simulates process being scheduled and completed by the process schedulers. Meanwhile, separate threads generate timer and I/O interrupts to test asynchronous event handling and process coordination. The system also runs a demo CPU program to verify correct instruction execution, memory access, and cache behavior. Overall, this test assesses whether all system components—CPU, memory hierarchy, interrupt controller, and process handling are work together smoothly to emulate a functioning
multitasking operating system environment.

\begin{lstlisting}[language=MyC, escapechar=\$, numbers=none]
	static void reset_cpu_and_memory(void) {
		memset(&THE_CPU, 0, sizeof(Cpu));
		free_memory();
		init_memory(CACHE_WRITE_THROUGH);
		set_current_process(SYSTEM_PROCESS_ID);
	}
	
	// ============================================
	// CPU Initialization Tests
	// ============================================
	
	TEST_CASE(CPU, InitializationSetsPC) {
		reset_cpu_and_memory();
		init_cpu(0x1000);
		ASSERT_EQ(HW_REGISTER(PC), 0x1000);
	}
	
	TEST_CASE(CPU, InitializationSetsZeroFlag) {
		reset_cpu_and_memory();
		init_cpu(0x2000);
		ASSERT_TRUE(HW_REGISTER(FLAGS) & F_ZERO);
	}
	
	TEST_CASE(CPU, InitializationClearsRegisters) {
		reset_cpu_and_memory();
		init_cpu(0x3000);
		for (int i = 1; i < GP_REG_COUNT; i++) {
			ASSERT_EQ(GP_REGISTER(i), 0);
		}
	}
	
	TEST_CASE(CPU, InitializationSetsZeroRegister) {
		reset_cpu_and_memory();
		init_cpu(0x4000);
		ASSERT_EQ(GP_REGISTER(REG_ZERO), 0);
	}
	
	// ============================================
	// Fetch Tests
	// ============================================
	
	TEST_CASE(CPU, FetchLoadsInstructionFromMemory) {
		reset_cpu_and_memory();
		uint32_t test_addr = 0x1000;
		uint32_t test_instruction = 0x12345678;
		
		write_word(test_addr, test_instruction);
		init_cpu(test_addr);
		fetch();
		
		ASSERT_EQ(HW_REGISTER(IR), test_instruction);
	}
	
	TEST_CASE(CPU, FetchIncrementsPC) {
		reset_cpu_and_memory();
		uint32_t start_addr = 0x2000;
		write_word(start_addr, 0xAAAAAAAA);
		
		init_cpu(start_addr);
		fetch();
		
		ASSERT_EQ(HW_REGISTER(PC), start_addr + 4);
	}
	
	TEST_CASE(CPU, FetchMultipleInstructions) {
		reset_cpu_and_memory();
		uint32_t base_addr = 0x3000;
		
		write_word(base_addr, 0x11111111);
		write_word(base_addr + 4, 0x22222222);
		write_word(base_addr + 8, 0x33333333);
		
		init_cpu(base_addr);
		
		fetch();
		ASSERT_EQ(HW_REGISTER(IR), 0x11111111);
		ASSERT_EQ(HW_REGISTER(PC), base_addr + 4);
		
		fetch();
		ASSERT_EQ(HW_REGISTER(IR), 0x22222222);
		ASSERT_EQ(HW_REGISTER(PC), base_addr + 8);
		
		fetch();
		ASSERT_EQ(HW_REGISTER(IR), 0x33333333);
		ASSERT_EQ(HW_REGISTER(PC), base_addr + 12);
	}
	
	// ============================================
	// Register Access Tests
	// ============================================
	
	TEST_CASE(CPU, GeneralPurposeRegisterReadWrite) {
		reset_cpu_and_memory();
		GP_REGISTER(REG_T0) = 0xDEADBEEF;
		ASSERT_EQ(GP_REGISTER(REG_T0), 0xDEADBEEF);
	}
	
	TEST_CASE(CPU, AllGPRegistersIndependent) {
		reset_cpu_and_memory();
		for (int i = 1; i < GP_REG_COUNT; i++) {
			GP_REGISTER(i) = i * 0x11;
		}
		for (int i = 1; i < GP_REG_COUNT; i++) {
			ASSERT_EQ(GP_REGISTER(i), (uint32_t)(i * 0x11));
		}
	}
	
	TEST_CASE(CPU, HardwareRegisterAccess) {
		reset_cpu_and_memory();
		HW_REGISTER(PC) = 0x5000;
		HW_REGISTER(IR) = 0xABCDEF12;
		HW_REGISTER(FLAGS) = F_ZERO | F_CARRY;
		
		ASSERT_EQ(HW_REGISTER(PC), 0x5000);
		ASSERT_EQ(HW_REGISTER(IR), 0xABCDEF12);
		ASSERT_EQ(HW_REGISTER(FLAGS), (uint32_t)(F_ZERO | F_CARRY));
	}
	
	TEST_CASE(CPU, StackPointerRegister) {
		reset_cpu_and_memory();
		GP_REGISTER(REG_SP) = 0x7FFFFFFC;
		ASSERT_EQ(GP_REGISTER(REG_SP), 0x7FFFFFFC);
		
		GP_REGISTER(REG_SP) -= 4;
		ASSERT_EQ(GP_REGISTER(REG_SP), 0x7FFFFFF8);
	}
	
	TEST_CASE(CPU, ReturnAddressRegister) {
		reset_cpu_and_memory();
		GP_REGISTER(REG_RA) = 0x1234;
		ASSERT_EQ(GP_REGISTER(REG_RA), 0x1234);
	}
	
	// ============================================
	// Flag Tests
	// ============================================
	
	TEST_CASE(CPU, ZeroFlagSetAndClear) {
		reset_cpu_and_memory();
		HW_REGISTER(FLAGS) = 0;
		HW_REGISTER(FLAGS) |= F_ZERO;
		ASSERT_TRUE(HW_REGISTER(FLAGS) & F_ZERO);
		
		HW_REGISTER(FLAGS) &= ~F_ZERO;
		ASSERT_TRUE(!(HW_REGISTER(FLAGS) & F_ZERO));
	}
	
	TEST_CASE(CPU, CarryFlagSetAndClear) {
		reset_cpu_and_memory();
		HW_REGISTER(FLAGS) = 0;
		HW_REGISTER(FLAGS) |= F_CARRY;
		ASSERT_TRUE(HW_REGISTER(FLAGS) & F_CARRY);
		
		HW_REGISTER(FLAGS) &= ~F_CARRY;
		ASSERT_TRUE(!(HW_REGISTER(FLAGS) & F_CARRY));
	}
	
	TEST_CASE(CPU, OverflowFlagSetAndClear) {
		reset_cpu_and_memory();
		HW_REGISTER(FLAGS) = 0;
		HW_REGISTER(FLAGS) |= F_OVERFLOW;
		ASSERT_TRUE(HW_REGISTER(FLAGS) & F_OVERFLOW);
		
		HW_REGISTER(FLAGS) &= ~F_OVERFLOW;
		ASSERT_TRUE(!(HW_REGISTER(FLAGS) & F_OVERFLOW));
	}
	
	TEST_CASE(CPU, MultipleFlagsSimultaneous) {
		reset_cpu_and_memory();
		HW_REGISTER(FLAGS) = F_ZERO | F_CARRY | F_OVERFLOW;
		
		ASSERT_TRUE(HW_REGISTER(FLAGS) & F_ZERO);
		ASSERT_TRUE(HW_REGISTER(FLAGS) & F_CARRY);
		ASSERT_TRUE(HW_REGISTER(FLAGS) & F_OVERFLOW);
	}
	
	TEST_CASE(CPU, ClearAllFlags) {
		reset_cpu_and_memory();
		HW_REGISTER(FLAGS) = F_ZERO | F_CARRY | F_OVERFLOW;
		HW_REGISTER(FLAGS) = 0;
		
		ASSERT_EQ(HW_REGISTER(FLAGS), 0);
	}
	
	// ============================================
	// HI/LO Register Tests
	// ============================================
	
	TEST_CASE(CPU, HIRegisterAccess) {
		reset_cpu_and_memory();
		HW_REGISTER(HI) = 0x12345678;
		ASSERT_EQ(HW_REGISTER(HI), 0x12345678);
	}
	
	TEST_CASE(CPU, LORegisterAccess) {
		reset_cpu_and_memory();
		HW_REGISTER(LO) = 0xABCDEF00;
		ASSERT_EQ(HW_REGISTER(LO), 0xABCDEF00);
	}
	
	TEST_CASE(CPU, HILOIndependentAccess) {
		reset_cpu_and_memory();
		HW_REGISTER(HI) = 0x11111111;
		HW_REGISTER(LO) = 0x22222222;
		
		ASSERT_EQ(HW_REGISTER(HI), 0x11111111);
		ASSERT_EQ(HW_REGISTER(LO), 0x22222222);
	}
	
	// ============================================
	// Fetch-Execute Cycle Integration Tests
	// ============================================
	
	TEST_CASE(CPU, SimpleFetchExecuteCycle) {
		reset_cpu_and_memory();
		uint32_t addr = 0x1000;
		
		// NOP instruction (sll $zero, $zero, 0)
		uint32_t nop = 0x00000000;
		write_word(addr, nop);
		
		init_cpu(addr);
		fetch();
		execute();
		
		ASSERT_EQ(HW_REGISTER(PC), addr + 4);
	}
	
	TEST_CASE(CPU, ExecutePreservesNonTargetRegisters) {
		reset_cpu_and_memory();
		
		// Set up registers
		GP_REGISTER(REG_T0) = 0xAAAAAAAA;
		GP_REGISTER(REG_T1) = 0xBBBBBBBB;
		GP_REGISTER(REG_T2) = 0xCCCCCCCC;
		
		// Execute NOP
		HW_REGISTER(IR) = 0x00000000;
		execute();
		
		// T0, T1, T2 should be unchanged
		ASSERT_EQ(GP_REGISTER(REG_T0), 0xAAAAAAAA);
		ASSERT_EQ(GP_REGISTER(REG_T1), 0xBBBBBBBB);
		ASSERT_EQ(GP_REGISTER(REG_T2), 0xCCCCCCCC);
	}
	
	// ============================================
	// Memory Access Through CPU Tests
	// ============================================
	
	TEST_CASE(CPU, CPUMemoryReadWrite) {
		reset_cpu_and_memory();
		uint32_t addr = 0x2000;
		uint32_t value = 0x12345678;
		
		write_word(addr, value);
		ASSERT_EQ(read_word(addr), value);
	}
	
	TEST_CASE(CPU, InstructionFetchFromDifferentAddresses) {
		reset_cpu_and_memory();
		
		write_word(0x1000, 0x11111111);
		write_word(0x2000, 0x22222222);
		write_word(0x3000, 0x33333333);
		
		init_cpu(0x1000);
		fetch();
		ASSERT_EQ(HW_REGISTER(IR), 0x11111111);
		
		HW_REGISTER(PC) = 0x2000;
		fetch();
		ASSERT_EQ(HW_REGISTER(IR), 0x22222222);
		
		HW_REGISTER(PC) = 0x3000;
		fetch();
		ASSERT_EQ(HW_REGISTER(IR), 0x33333333);
	}
	
	// ============================================
	// Register Naming Tests
	// ============================================
	
	TEST_CASE(CPU, ArgumentRegisters) {
		reset_cpu_and_memory();
		GP_REGISTER(REG_A0) = 1;
		GP_REGISTER(REG_A1) = 2;
		GP_REGISTER(REG_A2) = 3;
		GP_REGISTER(REG_A3) = 4;
		
		ASSERT_EQ(GP_REGISTER(REG_A0), 1);
		ASSERT_EQ(GP_REGISTER(REG_A1), 2);
		ASSERT_EQ(GP_REGISTER(REG_A2), 3);
		ASSERT_EQ(GP_REGISTER(REG_A3), 4);
	}
	
	TEST_CASE(CPU, ReturnValueRegisters) {
		reset_cpu_and_memory();
		GP_REGISTER(REG_V0) = 0x100;
		GP_REGISTER(REG_V1) = 0x200;
		
		ASSERT_EQ(GP_REGISTER(REG_V0), 0x100);
		ASSERT_EQ(GP_REGISTER(REG_V1), 0x200);
	}
	
	TEST_CASE(CPU, TemporaryRegisters) {
		reset_cpu_and_memory();
		for (int i = REG_T0; i <= REG_T7; i++) {
			GP_REGISTER(i) = i * 0x10;
		}
		for (int i = REG_T0; i <= REG_T7; i++) {
			ASSERT_EQ(GP_REGISTER(i), (uint32_t)(i * 0x10));
		}
	}
	
	TEST_CASE(CPU, SavedRegisters) {
		reset_cpu_and_memory();
		for (int i = REG_S0; i <= REG_S7; i++) {
			GP_REGISTER(i) = i * 0x100;
		}
		for (int i = REG_S0; i <= REG_S7; i++) {
			ASSERT_EQ(GP_REGISTER(i), (uint32_t)(i * 0x100));
		}
	}
	
	// ============================================
	// Edge Cases
	// ============================================
	
	TEST_CASE(CPU, PCWrapAround) {
		reset_cpu_and_memory();
		HW_REGISTER(PC) = 0xFFFFFFFC;
		// This would wrap on increment, but we test the value itself
		ASSERT_EQ(HW_REGISTER(PC), 0xFFFFFFFC);
	}
	
	TEST_CASE(CPU, MaximumRegisterValue) {
		reset_cpu_and_memory();
		GP_REGISTER(REG_T0) = 0xFFFFFFFF;
		ASSERT_EQ(GP_REGISTER(REG_T0), 0xFFFFFFFF);
	}
	
	TEST_CASE(CPU, MinimumRegisterValue) {
		reset_cpu_and_memory();
		GP_REGISTER(REG_T0) = 0x00000000;
		ASSERT_EQ(GP_REGISTER(REG_T0), 0x00000000);
	}
	
	// ============================================
	// State Consistency Tests
	// ============================================
	
	TEST_CASE(CPU, RepeatedInitialization) {
		reset_cpu_and_memory();
		
		init_cpu(0x1000);
		ASSERT_EQ(HW_REGISTER(PC), 0x1000);
		
		init_cpu(0x2000);
		ASSERT_EQ(HW_REGISTER(PC), 0x2000);
		
		init_cpu(0x3000);
		ASSERT_EQ(HW_REGISTER(PC), 0x3000);
	}
	
	TEST_CASE(CPU, StateAfterMultipleOperations) {
		reset_cpu_and_memory();
		
		init_cpu(0x1000);
		GP_REGISTER(REG_T0) = 42;
		HW_REGISTER(FLAGS) = F_ZERO;
		
		ASSERT_EQ(HW_REGISTER(PC), 0x1000);
		ASSERT_EQ(GP_REGISTER(REG_T0), 42);
		ASSERT_TRUE(HW_REGISTER(FLAGS) & F_ZERO);
	}
	
	// ============================================
	// MAR and MBR Tests
	// ============================================
	
	TEST_CASE(CPU, MemoryAddressRegister) {
		reset_cpu_and_memory();
		HW_REGISTER(MAR) = 0x5000;
		ASSERT_EQ(HW_REGISTER(MAR), 0x5000);
	}
	
	TEST_CASE(CPU, MemoryBufferRegister) {
		reset_cpu_and_memory();
		HW_REGISTER(MBR) = 0xDEADBEEF;
		ASSERT_EQ(HW_REGISTER(MBR), 0xDEADBEEF);
	}
	
	// ============================================
	// I/O Register Tests
	// ============================================
	
	TEST_CASE(CPU, IOAddressRegister) {
		reset_cpu_and_memory();
		HW_REGISTER(IO_AR) = 0x100;
		ASSERT_EQ(HW_REGISTER(IO_AR), 0x100);
	}
	
	TEST_CASE(CPU, IOBufferRegister) {
		reset_cpu_and_memory();
		HW_REGISTER(IO_BR) = 0xFF;
		ASSERT_EQ(HW_REGISTER(IO_BR), 0xFF);
	}
	
	static void reset_memory(void) {
		free_memory();
		init_memory(CACHE_WRITE_THROUGH);
		set_current_process(SYSTEM_PROCESS_ID);
	}
	
	static void reset_memory_write_back(void) {
		free_memory();
		init_memory(CACHE_WRITE_BACK);
		set_current_process(SYSTEM_PROCESS_ID);
	}
	
	// ============================================
	// Basic Read/Write Tests
	// ============================================
	
	TEST_CASE(Memory, ReadWriteByte) {
		reset_memory();
		write_byte(100, 0x42);
		ASSERT_EQ(read_byte(100), 0x42);
	}
	
	TEST_CASE(Memory, ReadWriteHalfword) {
		reset_memory();
		write_hword(200, 0x1234);
		ASSERT_EQ(read_hword(200), 0x1234);
	}
	
	TEST_CASE(Memory, ReadWriteWord) {
		reset_memory();
		write_word(300, 0x12345678);
		ASSERT_EQ(read_word(300), 0x12345678);
	}
	
	TEST_CASE(Memory, WriteBytePreservesOtherBytes) {
		reset_memory();
		write_word(400, 0xAABBCCDD);
		write_byte(401, 0xFF);
		ASSERT_EQ(read_word(400), 0xAABBFFDD);
	}
	
	TEST_CASE(Memory, WriteHalfwordPreservesOtherBytes) {
		reset_memory();
		write_word(500, 0x11223344);
		write_hword(502, 0xAABB);
		ASSERT_EQ(read_word(500), 0xAABB3344);
	}
	
	TEST_CASE(Memory, MultipleWordWrites) {
		reset_memory();
		for (uint32_t i = 0; i < 10; i++) {
			write_word(1000 + (i * 4), i * 100);
		}
		for (uint32_t i = 0; i < 10; i++) {
			ASSERT_EQ(read_word(1000 + (i * 4)), i * 100);
		}
	}
	
	// ============================================
	// Cache Tests (Write-Through)
	// ============================================
	
	TEST_CASE(Memory, CacheWriteThroughUpdatesRAM) {
		reset_memory();
		write_word(2000, 0xDEADBEEF);
		ASSERT_EQ(read_word(2000), 0xDEADBEEF);
	}
	
	TEST_CASE(Memory, CacheHitOnRepeatedRead) {
		reset_memory();
		write_word(3000, 0x12345678);
		uint32_t first = read_word(3000);
		uint32_t second = read_word(3000);
		ASSERT_EQ(first, second);
		ASSERT_EQ(first, 0x12345678);
	}
	
	TEST_CASE(Memory, CacheLineBoundary) {
		reset_memory();
		// Write at cache line boundary (64 bytes)
		write_word(0, 0xAAAAAAAA);
		write_word(64, 0xBBBBBBBB);
		ASSERT_EQ(read_word(0), 0xAAAAAAAA);
		ASSERT_EQ(read_word(64), 0xBBBBBBBB);
	}
	
	TEST_CASE(Memory, CacheWriteThroughConsistency) {
		reset_memory();
		uint32_t addr = 4000;
		write_word(addr, 0x11111111);
		write_word(addr, 0x22222222);
		write_word(addr, 0x33333333);
		ASSERT_EQ(read_word(addr), 0x33333333);
	}
	
	// ============================================
	// Cache Tests (Write-Back)
	// ============================================
	
	TEST_CASE(Memory, WriteBackDelaysRAMWrite) {
		reset_memory_write_back();
		write_word(5000, 0xFEEDFACE);
		ASSERT_EQ(read_word(5000), 0xFEEDFACE);
	}
	
	TEST_CASE(Memory, WriteBackMultipleWrites) {
		reset_memory_write_back();
		uint32_t addr = 6000;
		write_word(addr, 0x11111111);
		write_word(addr, 0x22222222);
		write_word(addr, 0x33333333);
		ASSERT_EQ(read_word(addr), 0x33333333);
	}
	
	TEST_CASE(Memory, WriteBackCacheLineEviction) {
		reset_memory_write_back();
		// Write enough data to force cache eviction
		for (uint32_t i = 0; i < 100; i++) {
			write_word(7000 + (i * 64), i);
		}
		for (uint32_t i = 0; i < 100; i++) {
			ASSERT_EQ(read_word(7000 + (i * 64)), i);
		}
	}
	
	// ============================================
	// Memory Allocation Tests
	// ============================================
	
	TEST_CASE(Memory, AllocateMemoryForProcess) {
		reset_memory();
		uint32_t addr = mallocate(1, 1024);
		ASSERT_TRUE(addr != UINT32_MAX);
	}
	
	TEST_CASE(Memory, AllocateMultipleBlocks) {
		reset_memory();
		uint32_t addr1 = mallocate(1, 512);
		uint32_t addr2 = mallocate(2, 512);
		ASSERT_TRUE(addr1 != UINT32_MAX);
		ASSERT_TRUE(addr2 != UINT32_MAX);
		ASSERT_TRUE(addr1 != addr2);
	}
	
	TEST_CASE(Memory, AllocateZeroSizeFails) {
		reset_memory();
		uint32_t addr = mallocate(1, 0);
		ASSERT_EQ(addr, UINT32_MAX);
	}
	
	TEST_CASE(Memory, FreeAllocatedMemory) {
		reset_memory();
		uint32_t addr = mallocate(1, 1024);
		ASSERT_TRUE(addr != UINT32_MAX);
		liberate(1);
		// Should be able to allocate same size again
		uint32_t addr2 = mallocate(2, 1024);
		ASSERT_TRUE(addr2 != UINT32_MAX);
	}
	
	TEST_CASE(Memory, BestFitAllocation) {
		reset_memory();
		// Allocate and free to create fragmentation
		uint32_t addr1 = mallocate(1, 512);
		uint32_t addr2 = mallocate(2, 1024);
		uint32_t addr3 = mallocate(3, 512);
		
		ASSERT_TRUE(addr1 != UINT32_MAX);
		ASSERT_TRUE(addr2 != UINT32_MAX);
		ASSERT_TRUE(addr3 != UINT32_MAX);
		
		liberate(2); // Free middle block
		
		// Should fit in freed space
		uint32_t addr4 = mallocate(4, 512);
		ASSERT_TRUE(addr4 != UINT32_MAX);
	}
	
	TEST_CASE(Memory, AllocateAfterMultipleFrees) {
		reset_memory();
		uint32_t addr1 = mallocate(1, 256);
		uint32_t addr2 = mallocate(2, 256);
		uint32_t addr3 = mallocate(3, 256);
		
		liberate(1);
		liberate(2);
		liberate(3);
		(void)addr1;
		(void)addr2;
		(void)addr3;
		
		uint32_t addr4 = mallocate(4, 512);
		ASSERT_TRUE(addr4 != UINT32_MAX);
	}
	
	// ============================================
	// Process Isolation Tests
	// ============================================
	
	TEST_CASE(Memory, ProcessCanAccessOwnMemory) {
		reset_memory();
		uint32_t addr = mallocate(1, 1024);
		set_current_process(1);
		write_word(addr, 0x12345678);
		ASSERT_EQ(read_word(addr), 0x12345678);
	}
	
	TEST_CASE(Memory, SystemProcessCanAccessAll) {
		reset_memory();
		set_current_process(SYSTEM_PROCESS_ID);
		write_word(1000, 0xAAAAAAAA);
		ASSERT_EQ(read_word(1000), 0xAAAAAAAA);
	}
	
	TEST_CASE(Memory, ProcessCanAccessTextSegment) {
		reset_memory();
		set_current_process(0);
		uint32_t text_addr = TEXT_BASE;
		write_word(text_addr, 0x11223344);
		ASSERT_EQ(read_word(text_addr), 0x11223344);
	}
	
	TEST_CASE(Memory, ProcessCanAccessDataSegment) {
		reset_memory();
		set_current_process(0);
		uint32_t data_addr = DATA_BASE;
		write_word(data_addr, 0x55667788);
		ASSERT_EQ(read_word(data_addr), 0x55667788);
	}
	
	// ============================================
	// Edge Cases and Bounds Tests
	// ============================================
	
	TEST_CASE(Memory, ReadAtAddressZero) {
		reset_memory();
		write_word(0, 0xCAFEBABE);
		ASSERT_EQ(read_word(0), 0xCAFEBABE);
	}
	
	TEST_CASE(Memory, WriteReadSequentialAddresses) {
		reset_memory();
		for (uint32_t i = 0; i < 100; i++) {
			write_byte(10000 + i, (uint8_t)(i & 0xFF));
		}
		for (uint32_t i = 0; i < 100; i++) {
			ASSERT_EQ(read_byte(10000 + i), (uint8_t)(i & 0xFF));
		}
	}
	
	TEST_CASE(Memory, LargeDataTransfer) {
		reset_memory();
		uint32_t base = 20000;
		for (uint32_t i = 0; i < 256; i++) {
			write_word(base + (i * 4), i * 0x11111111);
		}
		for (uint32_t i = 0; i < 256; i++) {
			ASSERT_EQ(read_word(base + (i * 4)), i * 0x11111111);
		}
	}
	
	TEST_CASE(Memory, AlternatingReadWrite) {
		reset_memory();
		uint32_t addr = 30000;
		write_word(addr, 0x12345678);
		ASSERT_EQ(read_word(addr), 0x12345678);
		write_word(addr, 0xAABBCCDD);
		ASSERT_EQ(read_word(addr), 0xAABBCCDD);
		write_word(addr, 0xFFEEDDCC);
		ASSERT_EQ(read_word(addr), 0xFFEEDDCC);
	}
	
	// ============================================
	// Endianness Tests
	// ============================================
	
	TEST_CASE(Memory, LittleEndianByteOrder) {
		reset_memory();
		write_word(40000, 0x12345678);
		ASSERT_EQ(read_byte(40000), 0x78);
		ASSERT_EQ(read_byte(40001), 0x56);
		ASSERT_EQ(read_byte(40002), 0x34);
		ASSERT_EQ(read_byte(40003), 0x12);
	}
	
	TEST_CASE(Memory, HalfwordEndianness) {
		reset_memory();
		write_word(41000, 0xAABBCCDD);
		ASSERT_EQ(read_hword(41000), 0xCCDD);
		ASSERT_EQ(read_hword(41002), 0xAABB);
	}
	
	// ============================================
	// Cache Statistics Tests
	// ============================================
	
	TEST_CASE(Memory, CacheStatsInitiallyZero) {
		reset_memory();
		// Just verify it doesn't crash
		print_cache_stats();
		ASSERT_TRUE(1);
	}
	
	TEST_CASE(Memory, ReadGeneratesCacheActivity) {
		reset_memory();
		for (int i = 0; i < 10; i++) {
			read_word(50000 + (i * 4));
		}
		// Cache should have some activity
		ASSERT_TRUE(1);
	}
	
	// ============================================
	// Memory Coalescing Tests
	// ============================================
	
	TEST_CASE(Memory, CoalesceAdjacentFreeBlocks) {
		reset_memory();
		uint32_t addr1 = mallocate(1, 256);
		uint32_t addr2 = mallocate(2, 256);
		
		ASSERT_TRUE(addr1 != UINT32_MAX);
		ASSERT_TRUE(addr2 != UINT32_MAX);
		
		liberate(1);
		liberate(2);
		
		// Should be able to allocate larger block
		uint32_t addr3 = mallocate(3, 512);
		ASSERT_TRUE(addr3 != UINT32_MAX);
	}
	
	// ============================================
	// Stress Tests
	// ============================================
	
	TEST_CASE(Memory, ManyAllocationsAndFrees) {
		reset_memory();
		for (int i = 0; i < 20; i++) {
			uint32_t addr = mallocate(i, 128);
			ASSERT_TRUE(addr != UINT32_MAX);
			if (i % 2 == 0) {
				liberate(i);
			}
		}
	}
	
	TEST_CASE(Memory, InterleavedCacheOperations) {
		reset_memory();
		for (int i = 0; i < 50; i++) {
			uint32_t addr = 60000 + (i * 8);
			write_word(addr, i);
			uint32_t val = read_word(addr);
			ASSERT_EQ(val, (uint32_t)i);
		}
	}
	
	static uint32_t make_i_instruction(uint32_t opcode, uint32_t rs, uint32_t rt,
	uint32_t immediate) {
		return (opcode & 0x3F) << OPCODE_SHIFT | (rs & 0x1F) << RS_SHIFT |
		(rt & 0x1F) << RT_SHIFT | (immediate & 0xFFFF);
	}
	
	static void reset_cpu_state(void) {
		memset(&THE_CPU, 0, sizeof(THE_CPU));
	}
	
	
	static void reset_cpu_and_memory(void) {
		reset_cpu_state();
		free_memory();
		init_memory(CACHE_WRITE_THROUGH);
		set_current_process(SYSTEM_PROCESS_ID);
	}
	
	static inline uint32_t mask_reg_index(uint32_t reg) {
		return (uint32_t)reg & 0x1F;
	}
	
	static inline int32_t read_gpr(uint32_t reg) {
		uint32_t idx = mask_reg_index(reg);
		if (idx == REG_ZERO || idx >= GP_REG_COUNT) {
			return 0;
		}
		return THE_CPU.gp_registers[idx];
	}
	
	static inline void write_gpr(uint32_t reg, uint32_t value) {
		uint32_t idx = mask_reg_index(reg);
		if (idx == REG_ZERO || idx >= GP_REG_COUNT) {
			return;
		}
		THE_CPU.gp_registers[idx] = value;
	}
	
	TEST_CASE(ITypeImmediate, AddiAddsSignedImmediate) {
		reset_cpu_and_memory();
		write_gpr(REG_T0, 5);
		execute_instruction(make_i_instruction(OP_ADDI, REG_T0, REG_T1, 0xFFFD));
		ASSERT_EQ(read_gpr(REG_T1), 2);
	}
	
	TEST_CASE(ITypeImmediate, AddiuWrapsUnsignedResult) {
		reset_cpu_and_memory();
		write_gpr(REG_T0, (uint32_t)0xFFFFFFFF);
		execute_instruction(make_i_instruction(OP_ADDIU, REG_T0, REG_T1, 0x0001));
		ASSERT_EQ((uint32_t)read_gpr(REG_T1), (uint32_t)0x00000000);
	}
	
	TEST_CASE(ITypeImmediate, AndiZeroExtendsImmediate) {
		reset_cpu_and_memory();
		write_gpr(REG_T0, (uint32_t)0xF0F0FFFF);
		execute_instruction(make_i_instruction(OP_ANDI, REG_T0, REG_T1, 0x8001));
		ASSERT_EQ((uint32_t)read_gpr(REG_T1), (uint32_t)0x00008001);
	}
	
	TEST_CASE(ITypeImmediate, OriCombinesBits) {
		reset_cpu_and_memory();
		write_gpr(REG_T0, 0x00FF0000);
		execute_instruction(make_i_instruction(OP_ORI, REG_T0, REG_T1, 0x1234));
		ASSERT_EQ((uint32_t)read_gpr(REG_T1), (uint32_t)0x00FF1234);
	}
	
	TEST_CASE(ITypeImmediate, XoriFlipsBits) {
		reset_cpu_and_memory();
		write_gpr(REG_T0, 0xFFFF0000);
		execute_instruction(make_i_instruction(OP_XORI, REG_T0, REG_T1, 0x0F0F));
		ASSERT_EQ((uint32_t)read_gpr(REG_T1), (uint32_t)0xFFFF0F0F);
	}
	
	TEST_CASE(ITypeImmediate, SltiPerformsSignedComparison) {
		reset_cpu_and_memory();
		write_gpr(REG_T0, -2);
		execute_instruction(make_i_instruction(OP_SLTI, REG_T0, REG_T1, 0x0001));
		ASSERT_EQ(read_gpr(REG_T1), 1);
	}
	
	TEST_CASE(ITypeImmediate, SltiuPerformsUnsignedComparison) {
		reset_cpu_and_memory();
		write_gpr(REG_T0, (uint32_t)0x80000000);
		execute_instruction(make_i_instruction(OP_SLTIU, REG_T0, REG_T1, 0xFFFF));
		ASSERT_EQ(read_gpr(REG_T1), 1);
	}
	
	TEST_CASE(ITypeImmediate, LuiLoadsUpperImmediate) {
		reset_cpu_and_memory();
		execute_instruction(make_i_instruction(OP_LUI, REG_ZERO, REG_T1, 0x1234));
		ASSERT_EQ((uint32_t)read_gpr(REG_T1), (uint32_t)0x12340000);
	}
	
	TEST_CASE(ITypeImmediate, LoadWordFetchesStoredValue) {
		reset_cpu_and_memory();
		write_gpr(REG_T0, 100);
		write_gpr(REG_T1, 0x12345678);
		execute_instruction(make_i_instruction(OP_SW, REG_T0, REG_T1, 0));
		write_gpr(REG_T1, 0);
		execute_instruction(make_i_instruction(OP_LW, REG_T0, REG_T1, 0));
		ASSERT_EQ((uint32_t)read_gpr(REG_T1), (uint32_t)0x12345678);
	}
	
	TEST_CASE(ITypeImmediate, LoadByteSignExtendsValue) {
		reset_cpu_and_memory();
		write_gpr(REG_T0, 120);
		write_gpr(REG_T1, 0x000000FF);
		execute_instruction(make_i_instruction(OP_SB, REG_T0, REG_T1, 0));
		write_gpr(REG_T2, 0);
		execute_instruction(make_i_instruction(OP_LB, REG_T0, REG_T2, 0));
		ASSERT_EQ((uint32_t)read_gpr(REG_T2), (uint32_t)0xFFFFFFFF);
	}
	
	TEST_CASE(ITypeImmediate, LoadByteUnsignedZeroExtends) {
		reset_cpu_and_memory();
		write_gpr(REG_T0, 140);
		write_gpr(REG_T1, 0x00000080);
		execute_instruction(make_i_instruction(OP_SB, REG_T0, REG_T1, 0));
		write_gpr(REG_T2, 0);
		execute_instruction(make_i_instruction(OP_LBU, REG_T0, REG_T2, 0));
		ASSERT_EQ((uint32_t)read_gpr(REG_T2), (uint32_t)0x00000080);
	}
	
	TEST_CASE(ITypeImmediate, LoadHalfSignExtendsValue) {
		reset_cpu_and_memory();
		write_gpr(REG_T0, 160);
		write_gpr(REG_T1, 0x0000F234);
		execute_instruction(make_i_instruction(OP_SH, REG_T0, REG_T1, 0));
		write_gpr(REG_T2, 0);
		execute_instruction(make_i_instruction(OP_LH, REG_T0, REG_T2, 0));
		ASSERT_EQ((uint32_t)read_gpr(REG_T2), (uint32_t)0xFFFFF234);
	}
	
	TEST_CASE(ITypeImmediate, LoadHalfUnsignedZeroExtends) {
		reset_cpu_and_memory();
		write_gpr(REG_T0, 180);
		write_gpr(REG_T1, 0x0000ABCD);
		execute_instruction(make_i_instruction(OP_SH, REG_T0, REG_T1, 0));
		write_gpr(REG_T2, 0);
		execute_instruction(make_i_instruction(OP_LHU, REG_T0, REG_T2, 0));
		ASSERT_EQ((uint32_t)read_gpr(REG_T2), (uint32_t)0x0000ABCD);
	}
	
	TEST_CASE(ITypeImmediate, StoreByteUpdatesSingleByte) {
		reset_cpu_and_memory();
		write_gpr(REG_T0, 200);
		write_gpr(REG_T1, 0x11223344);
		execute_instruction(make_i_instruction(OP_SW, REG_T0, REG_T1, 0));
		write_gpr(REG_T2, 0x000000AA);
		execute_instruction(make_i_instruction(OP_SB, REG_T0, REG_T2, 2));
		write_gpr(REG_T3, 0);
		execute_instruction(make_i_instruction(OP_LW, REG_T0, REG_T3, 0));
		ASSERT_EQ((uint32_t)read_gpr(REG_T3), (uint32_t)0x11AA3344);
	}
	
	TEST_CASE(ITypeImmediate, StoreHalfWritesTwoBytes) {
		reset_cpu_and_memory();
		write_gpr(REG_T0, 220);
		write_gpr(REG_T1, 0x00001234);
		execute_instruction(make_i_instruction(OP_SH, REG_T0, REG_T1, 0));
		write_gpr(REG_T2, 0);
		execute_instruction(make_i_instruction(OP_LW, REG_T0, REG_T2, 0));
		ASSERT_EQ((uint32_t)read_gpr(REG_T2), (uint32_t)0x00001234);
	}
	
	TEST_CASE(ITypeImmediate, BeqBranchesWhenEqual) {
		reset_cpu_and_memory();
		THE_CPU.hw_registers[PC] = 0x100;
		write_gpr(REG_T0, 7);
		write_gpr(REG_T1, 7);
		execute_instruction(make_i_instruction(OP_BEQ, REG_T0, REG_T1, 0x0002));
		ASSERT_EQ(THE_CPU.hw_registers[PC], (uint32_t)(0x100 + 4 + (2 << 2)));
	}
	
	TEST_CASE(ITypeImmediate, BneBranchesWhenNotEqual) {
		reset_cpu_and_memory();
		THE_CPU.hw_registers[PC] = 0x80;
		write_gpr(REG_T0, 1);
		write_gpr(REG_T1, 2);
		execute_instruction(make_i_instruction(OP_BNE, REG_T0, REG_T1, 0x0003));
		ASSERT_EQ(THE_CPU.hw_registers[PC], (uint32_t)(0x80 + 4 + (3 << 2)));
	}
	
\end{lstlisting}

\section{Conclusion}
\label{concl}

\subsection{Summary}

The operating system simulator represents a comprehensive exploration of core OS concepts through hands-on implementation and analysis. One of the key accomplishments of this project was the successful simulation of concurrent process execution, allowing multiple processes to be managed simultaneously while accurately modeling CPU behavior. Through advanced process simulation, the system demonstrates realistic handling of process states, execution flow, and synchronization challenges inherent in multitasking environments.

A major focus of the project was memory management and its tight integration with concurrency. By designing memory allocation and access mechanisms that operate correctly under concurrent workloads, the simulator highlights the complexities of shared resources, data consistency, and performance trade-offs. This integration reinforced the importance of coordination between memory systems and process execution in modern operating systems.

The simulator also implements several process scheduling strategies, emphasizing context switching under real-time and non-real-time constraints. Supporting multiple scheduling algorithms required careful handling of timing, preemption, and priority management. Through interrupt handling, the system models asynchronous events such as I/O completion and timer interrupts, further enhancing realism and demonstrating how operating systems maintain responsiveness and fairness in the presence of unpredictable events.

Finally, a comprehensive scheduling algorithm analysis module enables direct comparison of different scheduling techniques under varied workload scenarios. By collecting and evaluating metrics such as waiting time, turnaround time, response time, and cache behavior, the project provides insight into the strengths and weaknesses of each algorithm. Overall, this simulator strengthened understanding of how individual OS components interact as a cohesive system and deepened practical knowledge of performance evaluation, concurrency control, and system-level design.


\subsection{Key Learning Insights}

Working on this simulator deepened understanding of how operating system components interact under concurrent workloads. In particular, it highlighted the complexity of context switching, the impact of scheduling decisions on performance metrics, and the challenges of coordinating memory management with process execution. The project also reinforced the importance of empirical analysis when evaluating scheduling algorithms, demonstrating how different workloads can significantly influence fairness, responsiveness, and throughput. In addition, we learned how to make an assembler as well as the MIPS I architecture. 


\end{document}
